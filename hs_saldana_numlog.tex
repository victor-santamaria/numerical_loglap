\documentclass[11 pt]{article}  
\usepackage[utf8]{inputenc}
\usepackage{amsmath}
\usepackage{amsfonts}
\usepackage{amssymb}
\usepackage{graphicx}
\usepackage{mathrsfs}
\usepackage{upref,amsthm,amsxtra,exscale}
\usepackage{stmaryrd}
\usepackage{cite}
\usepackage[colorlinks=true,urlcolor=blue,
citecolor=red,linkcolor=blue,linktocpage,pdfpagelabels,
bookmarksnumbered,bookmarksopen]{hyperref}

\input fig4tex.tex


\newcommand\blue[1]{{\color{blue}\textbf{#1}}}

\newcommand\inter[1]{\llbracket #1\rrbracket}



\usepackage{fullpage}

\usepackage{subcaption}
\usepackage{caption}
\usepackage{cleveref}


\newtheorem{theorem}{Theorem}[section]
\newtheorem{corollary}[theorem]{Corollary}
\newtheorem{remark}[theorem]{Remark}
\newtheorem{lemma}[theorem]{Lemma}
\newtheorem{proposition}[theorem]{Proposition}
\newtheorem{definition}[theorem]{Definition}
\newtheorem{example}[theorem]{Example}
\numberwithin{equation}{section}

\usepackage{enumitem}

\def\N{\mathbb{N}}

\def\r{\mathbb{R}}
\def\diam{\operatorname{diam}}
\def\dist{\operatorname{dist}}
\def\rn{\mathbb{R}^N}
\def\rt{\mathbb{R}^3}
\def\z{\mathbb{Z}}
\def\zn{\mathbb{Z}^N}
\def\n{\mathbb{N}}
\def\cc{\mathbb{C}}
\def\eps{\varepsilon}
\def\rh{\rightharpoonup}
\def\io{\int_{\Omega}}
\def\irn{\int_{\r^N}}
\def\irt{\int_{\rt}}
\def\vp{\varphi}
\def\tilde{\widetilde}
\def\cD{\mathcal{D}}
\def\cI{\mathcal{I}}
\def\cJ{\mathcal{J}}
\def\cM{\mathcal{M}}
\def\cN{\mathcal{N}}
\def\cO{\mathcal{O}}
\def\cU{\mathcal{U}}
\def\cV{\mathcal{V}}
\def\cW{\mathcal{W}}

\newcommand{\norm}[2]{{\left\|#1\right\|}_{#2}}
\newcommand{\fl}[2]{(-d_x^2)^{#1}#2}
\newcommand{\rfl}[2]{A^{#1}_{\Omega}#2}
\newcommand{\hp}[1]{\hphantom{#1}}
\newcommand{\cns}{c_{N,s}}
\newcommand{\ccs}{c_{1,s}}
\newcommand{\ffl}[2]{(-d_x^{\,2})^{#1}#2}
\newcommand{\flh}[2]{\frac{1}{\Gamma(-s)}\int_0^{+\infty}\Big(e^{t\Delta}#2 - #2\Big)\frac{dt}{t^{1+#1}}}
\newcommand{\kernel}[1]{|x-y|^{#1}}
\newcommand{\dkj}{\delta_{kj}}
\newcommand{\intr}[1]{\underset{#1}{\int}}
\newcommand{\Do}[1]{D_{#1}}
\newcommand{\Hs}{H^s_0(\Omega)}
\newcommand{\ue}[1]{#1^{\,\varepsilon}}
\newcommand{\xHdot}[1]{\dot{H}^{#1}}
\newcommand{\ha}[2]{\mathbf{H}_{#1}^{#2}}
\newcommand{\lhi}{\mathcal{L}_i^h}
\newcommand{\NN}{\mathbb{N}}
\newcommand{\ZZ}{\mathbb{Z}}
\newcommand{\RR}{\mathbb{R}}
\newcommand{\CC}{\mathbb{C}}
\newcommand{\TT}{\mathbf{T}}
\newcommand\mesh{\mathfrak{M}}


\newcommand{\weH}[1]{\mathbb H^{#1}(\Omega;\ell)}


%Alberto's defs
\def\R{\mathbb{R}}
\def\S{\mathbb{S}}
\def\cP{\mathcal{P}}
\def\cM{\mathcal{M}}
\def\cL{\mathcal{L}}
\def\mH{\mathbb{H}}
\def\cE{\mathcal{E}}
\def\cC{\mathcal{C}}
\def\cH{\mathcal{H}}
\def\weakto{\rightharpoonup}
\def\d{\textnormal{d}}


\def\sideremark#1{\ifvmode\leavevmode\fi\vadjust{\vbox to0pt{\vss% the remark
 \hbox to 0pt{\hskip\hsize\hskip1em%                          will appear only
 \vbox{\hsize2.1cm\tiny\raggedright\pretolerance10000%          on the side
  \noindent #1\hfill}\hss}\vbox to15pt{\vfil}\vss}}}%
\newcommand{\edz}[1]{\sideremark{#1}}
    
    \usepackage{color}
\usepackage[dvipsnames]{xcolor}
\newcommand{\B}[1]{{\color{red} #1}}  %for paper content
\newcommand{\Bc}[1]{{\color{red}\textbf{#1}}}  %for comments
\newcommand{\SJ}[1]{{\color{green} #1}}  %for paper content
\newcommand{\SJc}[1]{{\color{green}\textbf{#1}}}  %for comments

\usepackage{tikz}
\usepackage{pgfplots}
\usepackage{pgfplotstable}
\usetikzlibrary{positioning}
\usepackage{booktabs}
%\usepackage{subfig}


\newcommand{\weakly}{\rightharpoonup}

\title{FEM for 1D-problems involving the logarithmic Laplacian: error estimates and numerical implementation}

\begin{document}

\maketitle

\section{Introduction}

In this paper we analyze the implementation of the finite element method (FEM) to one-dimensional nonlocal Dirichlet boundary value problems involving the logarithmic Laplacian such as
\begin{align}\label{P:intro}
    L_\Delta u = f \quad \text{ in }\Omega,\qquad u=0\quad \text{ on }\R\backslash \Omega,
\end{align}
where $\Omega$ is an open interval, $f$ belongs to a suitable (log-Hölder) space, and $L_\Delta$ denotes the logarithmic Laplacian given by
\begin{align}\label{LL}
     L_\Delta u(x)=\int_{\R}\frac{u(x)\chi_{B_1(x)}(y)-u(y)}{|x-y|}\, dy+\rho_1 u(x).
 \end{align}
 Here $\rho_1\approx-1.15$ is an explicit negative constant given in \eqref{constants}. The logarithmic Laplacian $L_\Delta$ is a pseudodifferential operator with Fourier symbol $2\ln|\xi|$, namely,
\begin{align*}
{\mathcal F}(L_\Delta \varphi)(\xi)=2\ln|\xi| {\mathcal F}(\varphi)(\xi)\qquad \text{ for all }\varphi\in C^\infty_c(\R),
\end{align*}
where ${\mathcal F}$ denotes the Fourier transform.  This operator appears naturally as a first order expansion of the (integral) fractional Laplacian $(-\Delta)^s$ (the pseudodifferential operator with Fourier symbol $|\xi|^{2s}$) as $s\to 0^+$; in particular, in \cite[Theorem 1.1]{CW19} it is shown that, for all $\varphi\in C^\infty_c(\R^N)$,
\begin{align*}
(-\Delta)^s\varphi = \varphi + sL_\Delta \varphi + o(s)\qquad \text{as $s\to 0^+$ in }L^p(\R) \text{ with }1<p\leq \infty.
\end{align*}

 Problems such as \eqref{P:intro} play a key role in the understanding of fractional equations as $s\to 0$. This asymptotic analysis is relevant, both in applications and for theoretical reasons.  In applications, for instance, several phenomena that are modeled with a fractional-type diffusion are optimized in some sense for small values of $s$, see \cite{Caffarelli17} and the references in \cite{HSS22}.  On the other hand,  the mathematical structures that appear in the limit as $s\to 0^+$ for linear and nonlinear fractional problems are highly nontrivial and interesting; for instance, we refer to \cite{CW19,LW21} for a study of the spectral properties of $L_\Delta$, to \cite{AS22,HSS22} for the analysis of some relevant nonlinear problems in this setting, and to \cite{CS22} for the regularity properties of solutions to \eqref{P:intro}, in particular, \cite[Theorem 1.1]{CS22} gives some conditions for the existence of classical solutions to \eqref{P:intro}.  We also mention \cite{JSW20}, where the operator $L_\Delta$ is used to characterize the derivative of the solution mapping of nonlocal problems involving the fractional Laplacian.

The problem \eqref{P:intro} has a variational structure (see \cite{CW19}), which allows the use of powerful functional analysis tools to explore its solvability. Nevertheless, the operator $L_\Delta$ exhibits several pathologies that make its study difficult. Here is a short list of some of them:
\begin{enumerate}
    \item It lacks good scaling properties (see the discussion in \cite{CS22}).
    \item For every bounded open set $U\subset \R$ there is $\lambda>0$ such that the problem \eqref{P:intro} does not have a solution for $f\equiv 1$ and $\Omega=\lambda U$, see \cite[Remark 5.9]{CS22}.
    \item The operator $L_\Delta$ does not satisfy the maximum principle in general. Nevertheless, positivity preserving properties do hold under additional geometric restrictions on the domain, see \cite[Theorem 1.8 and Corollary 1.9]{CW19}.
    \item Classical solutions of \eqref{P:intro} can be very irregular, in particular, they may not belong to Hölder spaces $C^\alpha(\Omega)$ for any $\alpha\in(0,1)$, see \cite[Remark 5.6]{CS22}.
    \item No explicit continuous solution of \eqref{P:intro} with $f>0$ is known. In particular, the torsion function, namely the solution of \eqref{P:intro} with $f\equiv 1$, is not known when $\Omega$ is a ball of radius $r>0$.
\end{enumerate}

 These obstacles justify the interest and importance of the analysis of the numerical approximation of solutions to \eqref{P:intro}.  Furthermore, we show that new phenomena appear and that new ideas are required to implement the finite element method in this setting.

In this work, we focus our analysis to dimension one. This simplified setting allows us to present the key ideas and new concepts in a more transparent way and it has several other advantages; the most relevant one, is that in 1D we can provide an \emph{explicit formula} for the stiffness matrix.  This makes the implementation of the FEM to be very straightforward and fast in just a few lines of code (available in GitHub).  We think that this provides the community with a compelling tool to understand the particular pathologies of logarithmic problems and to test and deduce new conjectures for these difficult and elusive problems. Furthermore, we believe that several of the ideas involved in the 1D analysis that we present can be generalized to dimensions 2 and 3, but certainly the higher-dimensional case is nontrivial and requires additional new ideas. See Section \ref{sec:o:p} for a discussion on the challenges in this direction and other open problems.

Our main result (see Theorem \ref{main:thm:intro} below) states that the FEM numerical approximation of the weak solution converges logarithmically; in particular, we show that the norm of the numerical error can be bounded in terms of $|\ln(h)|^{-\alpha}$, where $H$ is the discretization parameter and $\alpha\in(0,1)$. We also provide numerical evidence that suggests that this kind of logarithmic bounds are optimal, in the sense that no polynomial bound of the type $h^\beta$ for some $\beta>0$ can be expected in general in this setting.  This contrasts with the convergence rate $h^\frac{1}{2}|\ln h|$ observed for the fractional Laplacian for $s\in(0,\frac{1}{2})$ (see \cite[Theorem 3.31]{Bor17}) and the rate $h^2$ for the standard Laplacian (see \cite{???}).

To state our main result in more detail, let us introduce some notation. Let $L>0$,
\begin{align*}
\Omega:=(0,L),
\end{align*}
 $B_1(\Omega):=(-1,L+1)$. Following \cite{CW19}, consider the Hilbert space
\begin{align*}
\mH(\Omega):=\{u\in L^2(\Omega)\::\: \|u\|_{\mathbb H(\Omega)}<\infty \text{ and }u=0\text{ on }\R\backslash \Omega\},
\end{align*}
where $\|u\|^2_{\mathbb H(\Omega)}:={\mathcal E}(u,u)$ and ${\mathcal E}(\cdot,\cdot)$ is a scalar product on $\mathbb H(\Omega)$ given by
\begin{align*}
    {\mathcal E}(u,v)&=\int_{\R}\int_{B_1(x)}\frac{(u(x)-u(y))(v(x)-v(y))}{|x-y|}\, dy\, dx \qquad \text{ for }u,v\in \mH(\Omega).
\end{align*}
It is known that $\mathbb H(\Omega)\hookrightarrow L^2(\Omega)$ compact. Note that the space $\mH(\Omega)$ imposes very little regularity restrictions on its elements, and in particular step functions belong to $\mH(\Omega)$. As a consequence, this space does not allow a notion of trace for its elements.

As mentioned before, the problem \eqref{P:intro} has a variational structure. For $f\in L^2(\Omega)$, we say that $u\in \mH(\Omega)$ is a weak solution of \eqref{P:intro} if
\begin{equation*}
    \cE_{L}(u,v)=\int_{\Omega} f(x) v(x)\, dx \qquad \text{ for all } v\in  \mH(\Omega),
\end{equation*}
where $\cE_L(u,v)=\cE(u,v)+B(u,v)$ and
\begin{align}\label{eq:bilinear_B}
    B(u,v)=-\iint_{x,y\in\R\atop |x-y|\geq 1}\frac{u(x)\,v(y)}{|x-y|}dx dy+\rho_1\int_{\R}u v dx.
\end{align}
 The bilinear form $\cE_L$ is \emph{not} coercive in general and it does not induce a norm in $\mH(\Omega)$.

To implement the FEM in logarithmic problems, the regularity theory of weak solution of \eqref{P:intro} plays an important role. This kind of results face important challenges, due to the lack of good scaling properties of the operator $L_\Delta$; however, recently in \cite{CS22} the regularity of weak solutions in log-Hölder spaces has been studied.  Since these results play a fundamental role in our approach (in particular the estimate \eqref{eq:log_problem} below), it is worth to describe them in detail. Let $\ell:(0,\infty)\to (0,1]$ be the modulus of continuity given by
\begin{align}\label{ell:def}
\ell(\rho) := |\ln(\min(\rho_0,\rho))|^{-1}, \qquad\rho_0:=0.1.
\end{align}
Note that $\ell$ is a non-decreasing concave function with $\ell(0) := \lim_{r\to0^+}\ell(r)=0$. Given $\alpha\in(0,1)$, let
%
\begin{align*}
    \mathcal X^\alpha(\Omega)&:=\left\{u:\mathbb R\to \mathbb R:\|u\|_{X^\alpha(\Omega)}<+\infty \;\textnormal{ and $u=0$ in }\; \R\setminus \Omega\right\}, \\
    \mathcal Y(\Omega)&:=\{f:\Omega\to \mathbb R:\|f\|_{\mathcal Y(\Omega)}<+\infty\},
\end{align*}
%
where
%
\begin{align*}
    \|u\|_{\mathcal X^\alpha(\Omega)}&:=\sup_{x,y\in\mathbb R \atop x\neq y}\frac{|u(x)-u(y)|}{\ell^{\alpha}(|x-y|)}+\sup_{x,y\in\Omega \atop x\neq y}\ell^{1+\alpha}(d(x,y))\frac{|u(x)-u(y)|}{\ell^{1+\alpha}(|x-y|)}, \\
    \|f\|_{\mathcal Y(\Omega)}&:=\|f\|_{L^\infty(\Omega)}+\sup_{x,y\in\Omega \atop x\neq y}\ell^{2}(d(x,y))\frac{|\SJ{f}(x)-\SJ{f}(y)|}{\ell(|x-y  |)}, \\
    d(x)&:=\dist(x,\partial \Omega), \quad d(x,y):=\min(d(x),d(y)).
\end{align*}

The next theorem guarantees the existence of classical solutions of \eqref{P:intro}.

\begin{theorem}[Theorem 1.1 in \cite{CS22}]\label{eq:regularity}
    There is $\alpha=\alpha(\Omega)\in(0,1)$ such that exactly one of the following alternatives holds:
    \begin{enumerate}[label=\roman*)]
        \item For every $f\in\mathcal Y(\Omega)$ there exists a unique classical solution $u\in\mathcal X^{\alpha}(\Omega)$ of
        %
        \begin{equation}\label{eq:log_problem}
            L_{\Delta} u = f \;\textnormal{ in }\; \Omega, \quad u=0 \;\textnormal{ on } \R\setminus \Omega.
        \end{equation}
        %
        Moreover,
        %
        \begin{equation}\label{eq:regularity_classical}
            \|u\|_{\mathcal X^{\alpha}(\Omega)}\leq C\|f\|_{\mathcal Y(\Omega)}
        \end{equation}
        %
        for some constant $C>0$ only depending on $\Omega$.
        %
        \item There is a non-trivial classical solution $u\in X^{\alpha}(\Omega)$ of $L_{\Delta} u=0$.
    \end{enumerate}
\end{theorem}

As a final ingredient to state our main result, we introduce the notation of the 1D FEM.  Let $N\in\mathbb N$, $h=\frac{L}{N+1}$, $x_i:=ih$, and $\Omega_i:=(x_{i},x_{i+1})$ for $i=0,\ldots,N$. Note that $\overline{\Omega}=\cup_{i=0}^N \overline{\Omega_i}$. Consider the discrete space
%
\begin{equation}
    \mathcal V_h=\{v\in \mathcal C^0(\B{\Omega}): v=0 \textnormal{ in } \R\setminus \Omega, \;\; v|_{\Omega_i}\in \mathcal P_1, \;\;  i=0,\ldots,N\}.
\end{equation}
%
where $\mathcal P_1$ is the space of polynomials of degree less than or equal to one.
% Let $\varphi_i$ be the usual basis of shape functions $\varphi_i$ with compact support in each interval $[x_{i-1},x_{i+1}]$ defined as
% \begin{align*}
%     \varphi_0(x):=2\begin{cases}
%         \displaystyle \frac{x_{1}-x}{h} &x\in [x_0,x_{1}], \\
%         0 &\textnormal{elsewhere},
%     \end{cases} \qquad \varphi_{N+1}(x):=2\begin{cases}
%         \displaystyle \frac{x-x_{N}}{h} &x\in[x_{N},x_{N+1}], \\
%         0 &\textnormal{elsewhere,}
%     \end{cases}
% \end{align*}
% and, for $i=1,\ldots,N$,
% \begin{align*}
%     \varphi_i(x):=\begin{cases}
%         \displaystyle \frac{x-x_{i-1}}{h} &x\in[x_{i-1},x_i], \\
%         \displaystyle \frac{x_{i+1}-x}{h} &x\in [x_i,x_{i+1}], \\
%         0 &\textnormal{elsewhere.}
%     \end{cases}
% \end{align*}

We say that  $u_h\in \mathcal V_h$ is the \emph{finite element approximation} of the weak solution $u$ of \eqref{P:intro} if \begin{equation}\label{eq:discr_problem:intro}
    \cE_{L}(u_h,v)=\int_{\Omega} f(x) v(x)\, dx \qquad \text{ for all } v\in \mathcal V_h.
\end{equation}


 Our main result is the following.
 \begin{theorem}\label{main:thm:intro}
    Let $L>0$, $\Omega:=(0,L)$, and $f\in \mathcal{Y}(\Omega)$.  Assume that alternative $i)$ of \Cref{eq:regularity} holds for some $\alpha\in(0,1)$. Let $u$ be the (unique weak) solution of \eqref{eq:cont_problem} and let $u_h$ be its finite element approximation given by \eqref{eq:discr_problem:intro}. Then there are constants $h_0>0$, $C>0$, and $\alpha\in(0,1)$ such that
    \begin{equation*}
        \|u-u_h\|_{\mathbb H(\Omega)}\leq C\ell^{\alpha}(h)\|f\|_{\mathcal Y}\qquad \text{ for all }h\in(0,h_0),
    \end{equation*}
    where $\ell$ is the modulus of continuity given in \eqref{ell:def}.
 \end{theorem}
 Due to the nonlocal nature of $L_\Delta$, we cannot use the standard theory of FEM (note that the norms are not additive with respect to disjoint domains). Instead, for the proof of Theorem \ref{main:thm:intro}, we follow the general strategy implemented in \cite{Bor17}, where the FEM is used to approximate solutions of the fractional Laplacian and to show the convergence of the method.  The overall scheme is as follows: If $u$ is a weak solution that we wish to approximate with a finite element approximation, one defines first an \emph{interpolator} denoted by $I_h u$.  Then, via a localization of the norms (see Lemma \ref{lem:localization_enorm}) one can bound the global interpolation error $\|u-I_h u\|_{\mathbb H(\Omega)}$ (see Proposition \ref{eq:estab_H_c_infty}) in terms of a Hilbert space with logarithmic weights (see \eqref{H:weights:def}).  Then, with a Cea's-Lemma-type result (Lemma \ref{lem:cea_mg}), we can bound the norm $\|u-u_h\|_{\mathbb H(\Omega)}$ in terms of $\|u-I_h u\|_{\mathbb H(\Omega)}$.  Finally, the regularity estimate \eqref{eq:regularity_classical} comes into play to obtain the final estimate stated in Theorem \ref{main:thm:intro}.

 However, since the regularity theory for $L_\Delta$ is much less developed than that of $(-\Delta)^s$, key elements in the proofs in \cite{Bor17} have to be changed to fit into the logarithmic Laplacian setting.  For instance, the idea of \emph{trace} plays a key role in order to define the Scott-Zhang interpolator used in \cite{Bor17}. This trace is known to exist because, under suitable assumptions on the right-hand side $f$, regularity estimates for the fractional Laplacian imply that the corresponding weak solutions always belong to $H^{\frac{1}{2}+\eps}(\Omega)$ for some $\eps>0$ (see \cite[Theorem 2.3.6]{Bor17}), which is a Hilbert space with trace.  There is no such result for the operator $L_\Delta$.  Although Theorem \ref{eq:regularity} implies that weak solutions are log-Hölder continuous, it is not known if these solutions belong to a suitable Hilbert space with trace.

 As a consequence, we use a simpler interpolator: the projection to $\mathcal V_h$ in the $L^2$-sense (see \eqref{eq:interpolator}). Because of this, in our approach, it is important to consider two discontinuous elements in the finite element basis (see \eqref{eq:def_basis_ext}).  This might seem odd, in virtue of the Dirichlet boundary conditions, but in fact this is required in our analysis and numerical experiments show that the presence of these two discontinuous elements improves the numerical approximation of logarithmic problems.

 Finally, for the algorithmic implementation of the FEM, we also present a new perspective. Typically, in order to obtain the finite element approximation $u_h$ given by \eqref{eq:discr_problem:intro}, one needs to compute the so-called \emph{stiffness matrix}. For this, a finite element basis $(\varphi_i)_{i=0}^{N+1}$ of $\mathcal V_h$ is considered (see \eqref{eq:def_basis}-\eqref{eq:def_basis_ext}) and a matrix with entries
 \begin{align}\label{s:m:intro}
 \mathcal A_{h}^{L}=(\cE_L(\varphi_i,\varphi_j))_{i,j=0}^N
 \end{align}
 is calculated. Then, solving the linear algebraic problem
 \begin{align*}
\mathcal A_{h}^{L}\alpha = F,
 \end{align*}
 where $F=(f_i)_{i=0}^{N+1}$ and $f_i=\int_{\Omega}f(x)\varphi_i(x)\, dx$, one finds a vector $\alpha=(\alpha_i)_{i=0}^{N+1}$ and then
 \begin{align*}
     u_h(x):=\sum_{i=0}^{N+1}\alpha_i \varphi_i(x)
 \end{align*}
 is the finite element approximation and satisfies \eqref{eq:discr_problem:intro}. In this paper, instead of calculating \eqref{s:m:intro} (either directly or numerically), we use the stiffness matrix $\mathcal A_{h}^{s}$ for the fractional Laplacian $(-\Delta)^s$ and show that
 \begin{align*}
 \mathcal A_{h}^{L}=\partial_s \mathcal A_{h}^{s}|_{s=0},
 \end{align*}
 see Lemma \ref{lem:derivative:s:m}.  This is possible, since a closed formula for the stiffness matrix $\mathcal A_{h}^{s}$ has been obtained in \cite{BH17}.  Because the discontinuous base elements $\varphi_0$ and $\varphi_{N+1}$ were not considered in \cite{BH17}, here we had to calculate only the corresponding entries, see Section \ref{sec:s:m}.  We include a closed formula for $\mathcal A_{h}^{L}$ in Section \ref{sec:s:m:log}.  With this, the code implementation of the FEM for $L_\Delta$ is greatly simplified. We emphasize that the matrix $\mathcal A_{h}^{L}$ only depends on $h$ and not on the length $L$ of the interval $\Omega$.  Therefore, the formula for $\mathcal A_{h}^{L}$ applies to logarithmic problems on any interval.

To close this introduction, we discuss some numerical approximations.  First, we analyze the torsion function, namely, the solution of the equation
\begin{align}\label{t:p:intro}
 L_\Delta \tau = 1\quad \text{ in }(0,L),\qquad \tau=0\quad \text{ on }\R\backslash (-L,L)
\end{align}
for $L>0$.  In contrast to the case of the fractional Laplacian, the solution of \eqref{t:p:intro} does not have a closed formula. In fact, the numerical approximations show that the shape of the torsion functions changes drastically as the length of the domain $L>0$ increases, see Figure \ref{Fig:t:f}. In particular, for $L=0.2$ the torsion function is positive in $(0,L)$.  This is consistent with the known fact that the maximum principle for the logarithmic Laplacian only holds in small domains, see \cite[Corollary 1.9]{CW19}. For $L=2$ we observe that the maximum principle does not hold anymore and the solution is \emph{negative} in $(0,L)$ (which is sometimes referred to as an antimaximum principle) and for $L=16$ the torsion function exhibits oscillations and changes sign. These last two phenomena have not been described theoretically so far.

%%%%%%%%%%%%%%%
%%%%.        Figure       %%%
%%%%%%%%%%%%%%%
% DO NOT ERASE
\begin{figure}[h!]
	\centering
	%
	\subfloat[$L=0.2$]{
	\begin{tikzpicture}[scale=0.8]
	\begin{axis}[xlabel={$x$},xmin=0,xmax=0.2,legend pos=outer north east,
	legend plot pos=left, xtick={0,0.1,0.2}, ymin=-0.01,
	legend style={cells={anchor=west},draw=none}]
	%	    
	\pgfplotstableread{./Programas/num_results/sol-log_20-06-2023_12h20.org}\solution      
	      
	\addplot[very thick,color=blue] table[x,y] \solution; %\label{cost}
	
	\end{axis}
	%	
	\end{tikzpicture}
	}
	\subfloat[$L=2$]{
	\begin{tikzpicture}[scale=0.8]
	\begin{axis}[xlabel={$x$},xmin=0,xmax=2,legend pos=outer north east,
	legend plot pos=left, xtick={0,1,2}, ymax=0.01,
	legend style={cells={anchor=west},draw=none}]
	%	    
	\pgfplotstableread{./Programas/num_results/sol-log_20-06-2023_12h25.org}\solution      
	      
	\addplot[very thick,color=blue] table[x,y] \solution; %\label{cost}
	
	\end{axis}
	%	
	\end{tikzpicture}
	} 
	 \\ \bigskip
	\subfloat[$L=16$]{
	\begin{tikzpicture}[scale=0.8]
	\begin{axis}[xlabel={$x$},xmin=0,xmax=16,legend pos=outer north east,
	legend plot pos=left, xtick={0,8,16},
	legend style={cells={anchor=west},draw=none}]
	%	    
	\pgfplotstableread{./Programas/num_results/sol-log_20-06-2023_17h12.org}\solution      
	      
	\addplot[very thick,color=blue] table[x,y] \solution; %\label{cost}
	
	\end{axis}
	%	
	\end{tikzpicture}
	}
	\caption{Numerical aproximation of the solution to \eqref{t:p:intro} for different values of $L$.}
	\label{Fig:t:f}
\end{figure}

We have also tested our algorithm with some explicit solutions. We refer to Section \ref{sec:numerics} for more approximations and a discussion on the optimality of the logarithmic rate given in Theorem \ref{main:thm:intro}.


The paper is organized as follows....





\section{Notation}
Let
\begin{align}\label{constants}
\rho_1:=2\ln 2 + \psi(\tfrac{1}{2})-\gamma\approx -1.15,
\end{align}
where $\gamma=-\Gamma'(1)$ is the Euler-Mascheroni constant, and $\psi=\frac{\Gamma'}{\Gamma}$ is the Digamma function.



\subsection{Log-Hölder moduli of continuity}

The modulus of continuity $\ell$ (defined in \eqref{ell:def}) satisfies the following property.

\begin{lemma}[Semi-homogeneity]\label{prop1}
There is $c>0$ such that 
\[
\ell(\lambda)\leq c\frac{\ell(\lambda r)}{\ell(r)}\,\qquad \text{ for all }r,\lambda >0.
\]
\end{lemma}
For a proof, see \cite[Lemma 3.2]{CS22}.


\section{Variational setting}
Let $L>0$, $\Omega:=(0,L)$, $B_1(\Omega):=(-1,L+1)$. For $u\in \mH(\Omega)$, we have that
\begin{align*}
    \|u\|^2_{\mathbb H(\Omega)}&=\int_{\R}\int_{B_1(x)}\frac{|u(x)-u(y)|^2}{|x-y|}\, dy\, dx\\
    &=\int_{B_1(\Omega)}\int_{B_1(x)}\frac{|u(x)-u(y)|^2}{|x-y|}\, dy\, dx\\
    &=\int_\Omega\int_\Omega \frac{|u(x)-u(y)|^2}{|x-y|}\, dy\, dx+\int_{\Omega}h_\Omega(x)u(x)^2\, dx,
\end{align*}
where
\begin{align*}
    h_\Omega(x):=\left(\int_{B_1(x)\backslash \Omega}|x-y|^{-1}\, dy - \int_{\Omega\backslash B_1(x)}|x-y|^{-1}\, dy\right).
\end{align*}
%If $\Omega=(0,b)$ with $b\in(0,1)$, then
%\begin{align*}
%    h_\Omega(x)=-\ln(x(b-x)),\qquad x\in(0,b).
%\end{align*}


For any $\beta>0$, we denote by $\weH{\beta}$
%
\begin{align}\notag
    \weH{\beta}&=\left\{ u\in L^2(\Omega): \int_{\Omega}\int_{\Omega}\frac{|u(x)-u(y)|^2}{|x-y|\ell(|x-y|)^{\beta}}dydx<+\infty \right. \\
    &\hspace{3 cm}\left. \textnormal{and} \int_{\Omega}|u(x)|^2\ell^{-\beta}(\delta(x))dx <+\infty\right\}.\label{H:weights:def}
\end{align}

\section{Localization of the $\mathbb{H}$-norm in 1-d}

In the following, $C$ denotes possibly different positive constants that are independent of the parameter $h$ and $\inter{a,b}:=[a,b]\cap\mathbb{N}$ for any real numbers $a<b$. 

Let $\Omega=(0,L)$, $h=\frac{L}{N+1}$, and define the points $x_i:=ih$, so that
%
\begin{align}
0=x_0<x_1<\ldots<x_{N+1}=L.
\end{align}
We define $\Omega_i:=(x_{i},x_{i+1})$ for $i\in\inter{0,N}$ and we note that $\overline{\Omega}=\overline{\cup_{i\in\inter{0,N}}\Omega_i}$. For any $R>0$, we also define $B_{R}(\Omega):=\{x\in\R:\dist(x,\Omega)<R\}=(-R,L+R)$.

Unlike the usual fractional norm, we note that the $\mathbb H$-norm only has a finite range of interaction, indeed
%
\begin{align}\label{eq:norm_R_bound}
    \|u\|^2_{\mathbb H(\Omega)} &=\int_{B_1(\Omega)}\int_{B_1(x)}\frac{|u(x)-u(y)|^2}{|x-y|}\, dy\, dx
\end{align}
%
because if $\dist(x,\Omega)>1$ and $|x-y|<1$, then $y\not\in \Omega$.

\begin{lemma}\label{lem:localization_enorm}
Let $L>0$ be fixed and $N\in\mathbb N^*$ be large enough such that $h\in(0,\frac{1}{2})$. Then, there exists $C>0$ only depending on $L$ such that for any $u\in \mathbb{H}(\Omega)$, we have
   %
   \begin{align}\notag
    \|u\|^2_{\mathbb H(\Omega)} &\leq \sum_{i\in\inter{0,N}}\int_{\Omega_i}\int_{S_i}\frac{|u(x)-u(y)|^2}{|x-y|}dy\,dx+C\ell^{-1}(h)\sum_{i\in\inter{0,N}}\int_{\Omega_i}|u(x)|^2\,dx\\
    &\quad +C\sum_{i\in\{0,N\}}\int_{\Omega_i}|u(x)|^2\ell^{-1}(\delta(x))\,dx,
\end{align}
%
where $\ell(\cdot)$ as in \eqref{ell:def} and
%
\begin{align}\label{eq:def_Si}
    S_i:=\bigcup_{j\in\inter{0,N}\,:\,\overline{\Omega_j}\cap \overline{\Omega_i}\neq \emptyset}\Omega_j, \quad i\in\inter{0,N};
\end{align}
%
in particular,
\begin{align*}
S_0=(0,x_2)=(0,2h),\ S_{N}=(x_{N-1},x_{N+1})=(L-2h,L),\ S_{i}=(x_{i-1},x_{i+2})=(x_i-h,x_i+2h)
\end{align*}
for $i\in \inter{0,N-1}$.
\end{lemma}
%

\begin{proof}
We note that for $x\in B_1(\Omega)$, $B_1(x)\subset B_2(\Omega)$, therefore
%
\begin{align}\notag
    \|u\|^2&\leq \int_{B_2(\Omega)}\int_{B_2(\Omega)}\frac{|u(x)-u(y)|^2}{|x-y|}dy\, dx\\ \notag
    &=\int_{\Omega}\int_{\Omega}\frac{|u(x)-u(y)|^2}{|x-y|}dy\,dx+2\int_{\Omega}|u(x)|^2\int_{B_2(\Omega)\setminus B_h(\Omega)}\frac{1}{|x-y|}dy\,dx \\ \label{eq:part_init}
    &\quad +2\int_{\Omega}|u(x)|^2\int_{B_h(\Omega)\setminus \Omega}\frac{1}{|x-y|}dy\,dx =: I_1+2I_2+2I_3,
\end{align}
%
where we have used that $u=0$ in $\R\setminus \Omega$. 

We proceed to estimate each $I_i$. For $I_1$, we see that  
%
\begin{equation}
    I_1=\sum_{i\in\inter{0,N}}\left(\int_{\Omega_i}\int_{S_i}\frac{|u(x)-u(y)|^2}{|x-y|}\,dy\,dx+\int_{\Omega_i}\int_{D_i}\frac{|u(x)-u(y)|^2}{|x-y|}\,dy\,dx\right)
\end{equation}
%
where $D_i=\Omega\setminus S_i$. We will keep the first integrals as they are. For the second ones, we have by triangle inequality
%
\begin{align}\notag
    \int_{\Omega_i}&\int_{D_i} \frac{|u(x)-u(y)|^2}{|x-y|}dy\,dx \\
    &\leq 2\int_{\Omega_i}|u(x)|^2\int_{D_i}|x-y|^{-1}dy\,dx+2\int_{D_i}|u(y)|^2\int_{\Omega_i}|x-y|^{-1}dx\,dy=:2J_{i,1}+2J_{i,2}.
\end{align}
%
Following \cite[Prop. 1.2.24]{Bor17}, it is not difficult to see that $\sum_{i}{J_{i,1}}=\sum_{i}{J_{i,2}}$. Hence,
%
\begin{align}\label{eq:est_I1}
    I_1\leq \sum_{i\in\inter{0,N}}\left(\int_{\Omega_i}\int_{S_i}\frac{|u(x)-u(y)|^2}{|x-y|^N}dy\,dx+4 J_{i,1}\right).
\end{align}
%
To bound $J_{i,1}$, note that
%
\begin{align}\notag
    J_{i,1}&\leq \int_{\Omega_i}|u(x)|^2\int_{B_{L}(x)\setminus B_{\frac{h}{2}}(x)}|x-y|^{-1}dy\,dx \\\label{eq:est_Ij1}
    &\leq \int_{\Omega_i}|u(x)|^2\int_{\frac{h}{2}}^{L}\rho^{-1}d\rho\,dx=\left(-\log\left(\tfrac{h}{2L}\right)\right)\int_{\Omega_i}|u(x)|^2\,dx\leq C\ell^{-1} (h)\int_{\Omega_i}|u(x)|^2\,dx
\end{align}
%
for some $C>0$ independent of $h$ depending on $\Omega$ and where we have used that $h\in(0,1/2)$ in the last inequality. Thus, we have proved that
%
\begin{equation}\label{eq:est_I1_final}
    I_1\leq \sum_{i\in\inter{0,N}}\left(\int_{\Omega_i}\int_{S_i}\frac{|u(x)-u(y)|^2}{|x-y|^N}dy\,dx + C\ell^{-1}(h)\int_{\Omega_i}|u(x)|^2\,dx\right).
\end{equation}
%

For $I_2$, we see that it can be written as
%
\begin{equation}
    I_2=\sum_{i\in\inter{0,N}}\int_{\Omega_i}|u(x)|^2\int_{B_2(\Omega)\setminus B_{h}(\Omega)}|x-y|^{-1}dy\,dx
\end{equation}
%
Since $\dist(\Omega_i,B_2(\Omega)\setminus B_h(\Omega))>h$ for all $i\in\inter{0,N}$, we can argue as we did for \eqref{eq:est_Ij1} to obtain 
%
\begin{align}\label{eq:est_I2}
    I_2\leq C^\prime\ell^{-1}(h)\sum_{i\in\inter{0,N}}\int_{\Omega_i}|u(x)|^2\,dx
\end{align}
%
for some $C^\prime>0$ only depending on $\Omega$.

To estimate $I_3$, we begin by writing
%
\begin{align}\label{eq:est_I3_init}
    I_3=H_0+H_N+\sum_{i\in\inter{1,N-1}}H_i.
\end{align}
%
where $H_i=\int_{\Omega_i}|u(x)|^2\int_{B_h(\Omega)\setminus \Omega}|x-y|^{-1}dy\,dx$ for $i\in\inter{0,N}$. Note that for $i\in\inter{1,N-1}$, we have that $\dist(\Omega_i,B_h(\Omega)\setminus\Omega)>h$ so the second sum of \eqref{eq:est_I3_init} can be estimated as we did for $I_2$. On the other hand, for indices $i\in\{0,N\}$, we have that for some $R>0$ large enough
%
\begin{align*}
    H_i&\leq \int_{\Omega_i}|u(x)|^2\int_{B_R(0)\setminus B_{\delta(x)}(x)}|x-y|^{-1}dy\,dx \\
    &\leq \int_{\Omega_i}|u(x)|^2\left(-\log\left(\tfrac{\delta(x)}{R}\right)\right)\,dx \leq C \int_{\Omega_i}|u(x)|^2\ell^{-1}(\delta(x))\,dx
\end{align*}
%
where $C>0$ only depends on $\Omega$. Therefore,
%
\begin{align}\label{eq:est_I3}
    I_3\leq C \sum_{i\in\{0,N\}}\int_{\Omega_i}|u(x)|^2\ell^{-1}(\delta(x))\,dx + C\ell^{-1}(h)\sum_{\in\inter{1,N-1}}\int_{\Omega_i}|u(x)|^2\,dx
\end{align}
%
Finally, putting together \eqref{eq:est_I1_final}, \eqref{eq:est_I2}, and \eqref{eq:est_I3} gives the desired result. 
%
\end{proof}

\section{The discrete space $\mathcal V_h$}

We recall that $\overline{\Omega}=\overline{\cup_{i\in\inter{0,N}}\Omega_i}$ where $\Omega_i=(x_i,x_{i+1})$ for $i\in\inter{0,N}$. We consider the discrete space
%
\begin{equation}
    \mathcal V_h=\{v\in \mathcal C^0(\R): v=0 \textnormal{ in } \R\setminus \Omega, \;\; v|_{\Omega_i}\in \mathcal P_1, \;\;  i\in\inter{0,N}\}
\end{equation}
%
where $\mathcal P_1$ is the space of polynomials of degree less than or equal to one. 

Let $(\varphi_i)_{i\in\inter{0,N+1}}$ be the basis of shape functions $\varphi_i$ with compact support in each interval $[x_{i-1},x_{i+1}]$ defined as
%
\begin{align}\label{eq:def_basis}
    \textnormal{for $i\in\inter{1,N}$}:\quad \varphi_i(x)=\begin{cases}
        \displaystyle \frac{x-x_{i-1}}{h} &x\in[x_{i-1},x_i], \\
        \displaystyle \frac{x_{i+1}-x}{h} &x\in [x_i,x_{i+1}], \\
        0 &\textnormal{elsewhere}
    \end{cases}, \quad\textnormal{such that}\quad \varphi_i(x_j)=\delta_{ji}.
\end{align}
and
%
\begin{align}\label{eq:def_basis_ext}
    \varphi_0(x)=\B{\sqrt{2}}\begin{cases}
        \displaystyle \frac{x_{1}-x}{h} &x\in [x_0,x_{1}], \\
        0 &\textnormal{elsewhere},
    \end{cases} \qquad \varphi_{N+1}(x)=\B{\sqrt{2}}\begin{cases}
        \displaystyle \frac{x-x_{N}}{h} &x\in[x_{N},x_{N+1}], \\
        0 &\textnormal{elsewhere.}
    \end{cases}
\end{align}
%
We define the (quasi-)interpolator $I_h:L^2(\Omega)\to \mathcal V_h$ by
%
\begin{equation}\label{eq:interpolator}
I_h v(x)=\sum_{k\in\inter{0,N+1}}a_k \varphi_k(x),
\end{equation}
where
%
\begin{equation}\label{eq:def_coeff_ak}
  a_k:=\frac{1}{h}\int_{\Omega}v(x)\varphi_k(x)\,dx.
\end{equation}
%
Note that this definition with the $L^2$ scalar product is particularly useful since $v\in\mathbb H(\Omega)\subset L^2(\Omega)$ is not necessarily continuous at the grid points $x_i$.  Moreover, observe that $\varphi_0$ and $\varphi_{N+1}$ have a factor $\B{\sqrt{2}}$ and have a jump discontinuity at $0$ and at $N+1$ respectively.  This allows to have the important property that
\begin{align}\label{Ihc}
I_h 1 = 1\qquad \text{ in $\Omega$,}
\end{align}
namely, that the interpolation of a constant function in $\Omega$ is the constant itself.

%
\begin{lemma}\label{lem:bounds_aks}
Let $v\in \mathbb H(\Omega)$ and $a_k$ as in \eqref{eq:def_coeff_ak}. T
here is $C>0$ independent of $h$ such that
%
\begin{equation}\label{eq:est_ak_H}
    |a_k|\leq C h^{-1/2} |v|_{L^2(S_k)}, \quad\forall k\in\inter{0,N+1}
\end{equation}
%
with $S_k$ as defined in \eqref{eq:def_Si} and \blue{where $S_{N+1}:=S_N$}. If in addition, $v\in L^\infty(\Omega)$, we have
\begin{equation}\label{eq:est_ak_Linf}
    |a_k|\leq C |v|_{L^\infty(\Omega)}, \quad\forall k\in\inter{0,N+1}.
\end{equation}
\end{lemma}
%
\begin{proof}
Let $k\in\inter{0,N+1}$ and $v\in\mathbb H(\Omega)$ be an arbitrary given function. We have that
%
\begin{equation}
    |a_k|\leq \frac{1}{h}\int_{\Omega}|v||\varphi_k|dx.
\end{equation}
%
Note that $\textnormal{supp}\,\varphi_k\subset S_k$ for each $k\in\inter{0,N+1}$, \blue{where by convenience we have defined $S_{N+1}:=S_N$}. Using H\"older's inequality,
%
\begin{equation}
    |a_k|\leq h^{-1}|v|_{L^1(S_k)}\leq \sqrt{3}h^{-1/2}|v|_{L^2(S_k)}
\end{equation}
%
where we have used the fact that $|\varphi_k|_{L^\infty(\Omega)}=1$ for all $k\in\inter{0,N+1}$. Note that the right-hand side of the previous expression is finite since $\mathbb H(\Omega)\hookrightarrow L^2(\Omega)$. This proves \eqref{eq:est_ak_H}. 

To prove \eqref{eq:est_ak_Linf}, we use again that $|\varphi_k|_{L^\infty(\Omega)}=1$ for all $k\in\inter{0,N+1}$ to get
%
\begin{equation}
    |a_k|\leq \frac{1}{h}|v|_{L^\infty(\Omega)}\int_{S_k}|\varphi_k| dx \leq C|v|_{L^\infty(\Omega)}.
\end{equation}
%
This ends the proof.
\end{proof}

\begin{lemma}\label{lem:poincare_type}
Let $U\subset \R$ be an open bounded interval and for any $v\in C_c^\infty(U)$ we write $\overline v=\frac{1}{|U|}\int_{U}v$. Then,
%
\begin{equation}
    |v-\overline{v}|_{L^2(U)}^2\leq |v'|_{L^\infty(U)}|U|^3
\end{equation}
%
\end{lemma}
%
\begin{proof}
By definition and using Jensen's inequality we have
%
\begin{align}
    \int_{U}|v(x)-\overline{v}|^2dx=\frac{1}{|U|^2}\int_{U}\left|\int_{U}(v(x)-v(y))dy\right|^2dx \leq \frac{1}{|U|}\int_{U}\int_{U}|v(x)-v(y)|^2dy dx
\end{align}
%
whence
%
\begin{equation}
    \int_{U}|v(x)-\overline{v}|^2dx \leq \int_{U}\int_{U}\frac{|v(x)-v(y)|^2}{|x-y|}dy dx.
\end{equation}
%
By mean value theorem
%
\begin{equation}
    \int_{U}|v(x)-\overline{v}|^2dx 
    \leq |v'|_{L^\infty(U)}\int_{U}\int_{U}|x-y|dy dx
    \leq |v'|_{L^\infty(U)}|U|^3,
\end{equation}
%
since $|x-y|\leq |U|$ for all $x,y\in U$.
\end{proof}

\begin{lemma}\label{lem:inter_l_infty}
    Let $v\in C_c^\infty(\Omega)$. Then, $|I_h v|_{L^\infty(\Omega)}\leq C|v|_{L^\infty(\Omega)}$ for some $C>0$ uniform with respect to $h$.
\end{lemma}
%
\begin{proof}
Let $x\in \Omega$ be arbitrary. From definition \eqref{eq:interpolator} and estimate \eqref{eq:est_ak_Linf} we readily have
%
\begin{align*}
    |I_h v(x)|=\left|\sum_{k\in\inter{0,N+1}}a_k \varphi_k(x)\right| \leq C|v|_{L^\infty(\Omega)}\sum_{k\in\inter{0,N+1}}\varphi_k(x).
\end{align*}
%
By the definition of the basis functions, we note that 
%
\begin{equation*}
    \sum_{k\in\inter{0,N+1}}\varphi_k(x)\leq \B{\sqrt{2}} \quad \textnormal{ for any $x\in\Omega$}.
\end{equation*}
The result then follows. 
\end{proof}

\blue{Modificar la notación de este lema, posiblemente cambien algunas cosas abajo pero es solo cuestión de notación}

\begin{lemma}\label{lem:loc_norm_interp}
Let $i\in\inter{0,N}$, $v\in \mathbb H(\Omega)$, and $a_k$ as in \eqref{eq:def_coeff_ak}.  There is $C>0$ independent of $h$ such that
%
\begin{enumerate}[label=\upshape(\roman*)]
\item $\displaystyle \int_{\Omega_i}\int_{S_i}\frac{|I_hv(x)-I_hv(y)|^2}{|x-y|}dy dx\leq Ch\sum_{k\in \inter{i-1,i+2}} a_k^2$,
\item  $\displaystyle |I_h u|^2_{L^2(\Omega_i)}\leq Ch\sum_{k\in \inter{i-1,i+1}} a_k^2$,
\end{enumerate}
\end{lemma}
\begin{proof}
Let $I_i:=\{j\in\inter{0,N+1}:\varphi_j(x)-\varphi_j(y) \neq 0 \ \textnormal{for $x\in \Omega_i$ and $y\in S_i$}\}=\inter{i-1,i+2}$ and  $I^\prime_i:=\{j\in\inter{0,N+1}:\varphi_j(x)\neq 0 \ \textnormal{for $x\in \Omega_i$}\}=\inter{i-1,i+1}$ \blue{Me parece que solo es $\inter{i,i+1}$}. We begin by proving item (i). Given $i\in\inter{0,N}$, let $x\in \Omega_i$ and $y\in S_i$ such that $x\neq y$.  From definitions \eqref{eq:def_basis} and \eqref{eq:interpolator} it can be readily seen that
%
\begin{equation}\label{a1}
    I_hv(x)-I_hv(y)=\sum_{k\in I_i} a_k(\varphi_k(x)-\varphi_k(y)).
\end{equation}
%
Since $I_i$ has \blue{at most} 4 elements, squaring both sides of \eqref{a1} and using triangle inequality,
%
\begin{equation}\label{eq:est_interp_1}
    \int_{\Omega_i}\int_{S_i}\frac{|I_hv(x)- I_hv(y)|^2}{|x-y|} dy dx \leq C \sum_{k\in I_i} a_k^2\int_{\Omega_i}\int_{S_i} \frac{|\varphi_k(x)- \varphi_k(y)|^2}{|x-y|} dy dx.
\end{equation}
%

Noting that the functions $\varphi_k$ are Lipschitz continuous and verify that $|\varphi_k^\prime|\leq C/h$ for all $k\in\inter{0,N+1}$, we can use the mean value theorem to deduce
%
\begin{equation}\label{eq:est_interp_2}
    \int_{\Omega_i}\int_{S_i}\frac{|\varphi_k(x)-\varphi_k(y)|^2}{|x-y|}dy dx \leq \frac{C}{h^2}\int_{\Omega_i}\int_{S_i}|x-y|dy dx \leq Ch,
\end{equation}
%
since $|x-y|\leq 2h$ for all $x\in \Omega_i$ and $y\in S_i$. Combining \eqref{eq:est_interp_1} and \eqref{eq:est_interp_2} yields (i). For (ii), note that
\begin{align*}
|I_h u|^2_{L^2(\Omega_i)}
\leq \sum_{k\in \inter{i-1,i+1}}|a_k|^2\int_{\Omega_i}\varphi_k(x)^2\, dx
\leq \sum_{k\in \inter{i-1,i+1}}|a_k|^2|\Omega_i|=h\sum_{k\in \inter{i-1,i+1}}|a_k|^2.
\end{align*}
%
\end{proof}



\begin{lemma}\label{lem:stab_dif_interp}
Let $i\in\inter{0,N}$ and $v\in C_c^\infty(\Omega)$. There is $C>0$ independent of $h$ such that
\begin{align}\label{eq:approxim_estimate}
    \int_{\Omega_i}\int_{S_i}\frac{|(v-I_h v)(x)-(v-I_h v)(y)|^2}{|x-y|}dy dx \leq C h^3. 
\end{align}
\end{lemma}
%
\begin{proof}
Let $i\in\inter{0,N}$, $v\in C_c^\infty(\Omega)$, and $\overline {v}:=\frac{1}{5h}\int_{x_i-2h}^{x_i+3h}v\,dx$.
By \eqref{Ihc}, $v-I_h v=v-\overline{v}+I_h(\overline v - v)$, thus
%
\begin{align}\notag
    \int_{\Omega_i}&\int_{S_i}\frac{|(v-I_h v)(x)-(v-I_h v)(y)|^2}{|x-y|} dy dx \\ \notag
    &\leq 2\int_{\Omega_i}\int_{S_i}\frac{|(v-\overline{v})(x)-(v-\overline{v})(y)|^2}{|x-y|}dy dx + 2 \int_{\Omega_i}\int_{S_i}\frac{|I_h(\overline v-{v})(x)-I_h(\overline v-{v})(y)|^2}{|x-y|}dy dx \\ \label{eq:est_dif_inter_1}
    &=:2J_1+2J_2.
\end{align}
%

Using that $\overline v$ is a constant function, we can use mean value theorem to bound $J_1$ as
%
\begin{equation}\label{eq:est_J1}
    J_1 = \int_{\Omega_i}\int_{S_i}\frac{|v(x)-v(y)|^2}{|x-y|}dy dx \leq C \int_{S_i}\int_{S_i}|x-y|dy dx \leq Ch^3,
\end{equation}
since $|x-y|\leq 3h$ for $x,y\in S_i$.

On the other hand, we can use item (i) of \Cref{lem:loc_norm_interp} to $J_2$ and obtain
%
\begin{equation}\label{eq:est_J2}
    J_2 \leq Ch\sum_{k\in I_i} b_k^2,
\end{equation}
%
where $b_k:=\frac{1}{h}\int_{\Omega}(v-\overline v)\varphi_k dx$ and $I_i=\{j\in\inter{0,N}:\varphi_j(x)-\varphi_j(y) \neq 0 \ \textnormal{for $x\in \Omega_i$ and $y\in S_i$}\}=\inter{i-1,i+2}$. Since $|\varphi_k|\leq 1$, 
\begin{equation}\label{abk}
    |b_k|^2
    \leq \frac{1}{h^2}\left(\int_{S_k} |v-\overline v|dx\right)^2
    \leq \frac{C}{h}\int_{S_k}|v-\overline{v}|^2\, dx
    \leq \frac{C}{h}\int_{x_i-2h}^{x_i+3h}|v-\overline{v}|^2\, dx
\end{equation}
for $k\in I_i.$ Using \Cref{lem:poincare_type}, we can estimate the right-hand side of the above expression and obtain
%
\begin{equation}\label{eq:est_bks}
    |b_k|^2\leq Ch^2.
\end{equation}
%
Collecting estimates \eqref{eq:est_dif_inter_1}--\eqref{eq:est_bks} gives the desired result. This ends the proof. 
\end{proof}

\begin{remark}\label{rem:l2_stability}
By making minor adjustments to the proof of  \Cref{lem:stab_dif_interp}, we can use item (ii) of \Cref{lem:loc_norm_interp} to obtain the following $L^2$-stability estimate, for $i\in\inter{0,N}$,
%
\begin{equation*}
    |v-I_hv|_{L^2(\Omega_i)} \leq C h^3, \quad  v\in C_c^\infty(\Omega).
\end{equation*}
Indeed, let $i\in\inter{0,N}$, $U=({x_i-2h},{x_{i}+2h})$, and $\overline{v}:=\frac{1}{U}\int_U v\, dx$; then, by Lemmas \ref{lem:poincare_type} and \ref{lem:loc_norm_interp},
\begin{align}
|v-I_hv|^2_{L^2(\Omega_i)}
\leq C|v-\overline{v}|_{L^2(\Omega_i)}^2 +|I_h(v-\overline{v})|^2_{L^2(\Omega_i)}
\leq Ch^3+Ch\sum_{k\in \inter{i-1,i+1}}a_k^2,\label{a2}
\end{align}
where $a_k=\frac{1}{h}\int_{x_{k-1}}^{x_{k+1}} (v-\overline{v})\varphi_k\, dx$ and $|a_k|\leq \frac{1}{h}\int_{x_i-2h}^{x_{i}+2h}|v-\overline{v}|\, dx$. By Lemma \ref{lem:poincare_type},
\begin{equation}
    |v-\overline{v}|_{L^2(U)}^2\leq Ch^3.
\end{equation}
The claim now follows from \eqref{a2}.
\end{remark}

\begin{proposition}\label{eq:estab_H_c_infty}
For any $\alpha>0$, there exists $h_0>0$ only depending on $\alpha$ such that for any $h<h_0$ and any $v\in C_c^\infty(\Omega)$ we have
%
\begin{equation}\label{eq:rate_H_c_infty}
    \|v-I_h v\|_{\mathbb H(\Omega)}\leq C\ell^\alpha(h).
\end{equation}
%
\end{proposition}

\begin{proof}
Combining \eqref{eq:approxim_estimate} and \Cref{rem:l2_stability} with \Cref{lem:localization_enorm} we obtain
%
\begin{align*}
    \|v-I_h v\|_{\mathbb H(\Omega)}&\leq  C h^3(N+1)+Ch^3(N+1)\ell^{-1}(h)+C\sum_{i\in\{0,N\}}\int_{\Omega_i}|v-I_h v|^2\ell^{-1}(\delta(x))\,dx \\
    &\leq Ch^2+Ch^2\ell^{-1}(h)+C\sum_{i\in\{0,N\}}\int_{\Omega_i}|v-I_h v|^2\ell^{-1}(\delta(x))\,dx
\end{align*}
since $h=\frac{L}{N+1}$. Since $v\in C_c^\infty(\Omega)$, a direct computation and \Cref{lem:inter_l_infty} yield
%
\begin{align*}
    \|v-I_h v\|_{\mathbb H(\Omega)} &\leq Ch^2+Ch^2\ell^{-1}(h)+C\left(\|v\|_{L^\infty(\Omega)}^2+\|I_hv\|^2_{L^\infty(\Omega)}\right)\sum_{i\in\{0,N\}}\int_{\Omega_i}\ell^{-1}(\delta(x))\,dx \\
    & \leq Ch^2+Ch^2\ell^{-1}(h)+C\ell^{\alpha}(h)\sum_{i\in\{0,N\}}\int_{\Omega_i}\ell^{-1-\alpha}(\delta(x))\,dx.
\end{align*}
%
Using Lemma \ref{lem:ellbeta} and since $h\leq \ell^{\frac{1+\alpha}{2}}(h)$ for any $h\leq h_0$ for some $h_0>0$ only depending on $\alpha$, we obtain
%
\begin{align*}
    \|v-I_h v\|_{\mathbb H(\Omega)}
    & \leq C\ell^{1+\alpha}(h)+C\ell^{\alpha}(h)\leq C \ell^{\alpha}(h),
\end{align*}
which is the desired result.
\end{proof}

\begin{remark}
We shall note that even if the previous result is valid for any $\alpha>0$, there is a trade-off between the rate of convergence \eqref{eq:rate_H_c_infty} and the smallness of the parameter $h_0$. Indeed, the bigger $\alpha$ is, the smaller $h_0$ is. For instance, for $\alpha=1$, $h_0=(\log(2)+\log(5))^{-1}\approx 0.4342$ while for $\alpha=9$, $h_0\approx 3\times 10^{-6}$. Note however that the decay rate \eqref{eq:rate_H_c_infty} would still be logarithmic and there is not really much improvement in the speed of convergence.  
\end{remark}

We are in position to prove the main result of this section.

\begin{theorem}\label{density:thm}
    For any $v\in\mathbb H(\Omega)$ there exists $v_h\in\mathcal V_h$ such that $\|v-v_h\|_{\mathbb H(\Omega)}\to 0$ as $h\to 0$. 
\end{theorem}

\begin{proof}
%
From the definition of $\mathbb H(\Omega)$, it is clear that Lipschitz continuous functions which are zero on $\R\setminus \Omega$ belong to $\mathbb H(\Omega)$, hence $\mathcal V_h$ is a subspace of $\mathbb H(\Omega)$.

By Hilbert projection theorem, for every $v\in\mathbb H(\Omega)$ there exists a unique element $v_h\in\mathcal \mathcal V_h$ such that
%
\begin{equation}\label{eq:prop_infimo}
    \|v-v_h\|_{\mathbb H(\Omega)}=\inf_{w\in\mathcal V_h}\|v-w\|_{\mathbb H(\Omega)}. 
\end{equation}
%
Actually, $v_h=\Pi_h v$ where $\Pi_h$ is the orthogonal projection from $\mathbb H(\Omega)$ onto $\mathcal V_h$. Additionally, $\|\Pi_h\|_*=1$ where $\|\Pi_h\|_*:=\sup\{\|\Pi_hv\|_{\mathbb H(\Omega)}: v\in\mathbb{H},\;\; \|v\|_{\mathbb H(\Omega)}\leq 1\}$.

Let $\epsilon>0$. We recall that $C_c^\infty(\Omega)$ is dense in $\mathbb H(\Omega)$. Thus, for every $v\in\mathbb H(\Omega)$ there exists $w\in C_c^\infty(\Omega)$ such that 
%
\begin{equation}\label{eq:density}
    \|v-w\|_{\mathbb H(\Omega)}<\epsilon/3.
\end{equation}
%
By triangle inequality, we have
%
\begin{align}
    \|v-v_h\|_{\mathbb H(\Omega)}&\leq \|v-w\|_{\mathbb H(\Omega)}+\|w-\Pi_h w\|_{\mathbb H(\Omega)}+\|\Pi_h w-\Pi_h v\|_{\mathbb H(\Omega)} \\
    &\leq \frac{2\epsilon}{3}+\|w-I_hw\|_{\mathbb H(\Omega)}. 
\end{align}
%
where we have used that $\|\Pi_h\|_*=1$ and property \eqref{eq:prop_infimo}. Lastly, using \Cref{eq:estab_H_c_infty} with $h>0$ small enough, $\|w-I_hw\|_{\mathbb H(\Omega)}\leq \epsilon/3$ and therefore $\|v-v_h\|_{\mathbb H(\Omega)}\leq \epsilon$. This yields the claim.
\end{proof}


\section{Quasi-interpolation estimates}

\begin{lemma}\label{lem:poincare_type_weight}
Let $\alpha>0$, $U\subset \R$ be an open bounded interval, $v\in \mathbb H^{\alpha}(U;\ell)$, and $\overline v=\frac{1}{|U|}\int_{U}v\, dx$, then,
%
\begin{equation}
    |v-\overline{v}|_{L^2(U)}^2\leq \ell^{\alpha}(|U|)\int_{U}\int_{U}\frac{|v(x)-v(y)|^2}{|x-y|\ell^{\alpha}(|x-y|)}dy dx.
\end{equation}
%
\end{lemma}
%
\begin{proof}
By Jensen's inequality,
%
\begin{align*}
    \int_{U}|v(x)-\overline{v}|^2dx
    &=\frac{1}{|U|^2}\int_{U}\left|\int_{U}(v(x)-v(y))dy\right|^2dx \leq \frac{1}{|U|}\int_{U}\int_{U}|v(x)-v(y)|^2dy dx\\
&\leq \int_{U}\int_{U}\frac{|v(x)-v(y)|^2}{|x-y|}dy dx \leq \ell^{\alpha}(|U|)\int_{U}\int_{U}\frac{|v(x)-v(y)|^2}{|x-y|\ell^{\alpha}(|x-y|)}dy dx.
\end{align*}
%
This ends the proof. 
\end{proof}



\begin{lemma}\label{lem:stab_dif_interp_weight}
Let $\alpha>0$, $i\in\inter{0,N}$, $v\in \weH{\alpha}$, then
\begin{align}
    \int_{\Omega_i}\int_{S_i}\frac{|(v-I_h v)(x)-(v-I_h v)(y)|^2}{|x-y|}dy dx &\leq C  \ell^{\alpha}(h)\int_{T_i}\int_{T_{i}}\frac{|v(x)-v(y)|^2}{|x-y|\ell^{\alpha}(|x-y|)}dy dx, \label{eq:approxim_estimate_weight}\\
    |v-I_h v|_{L^2(\Omega_i)}^2&\leq C\ell^{\alpha}(h)\int_{T_{i}}\int_{T_{i}}\frac{|v(x)-v(y)|^2}{|x-y|\ell^{\alpha}(|x-y|)}dy dx,\label{a4}
\end{align}
where $T_i:=S_{i-1}\cup S_{i+1}$ \blue{and $S_{i}=\emptyset$ if $i=-1$ or $i={N+1}$}.
\end{lemma}
%
\begin{proof}
Let $\overline{v}:=\frac{1}{5h}\int_{T_{i}} v\, dx$. Note that $v-I_h v=v-\overline{v}+I_h(\overline v - v)$, thus
%
\begin{align}\notag
    \int_{\Omega_i}&\int_{S_i}\frac{|(v-I_h v)(x)-(v-I_h v)(y)|^2}{|x-y|} dy dx \\ \notag
    &\leq 2\int_{\Omega_i}\int_{S_i}\frac{|(v-\overline{v})(x)-(v-\overline{v})(y)|^2}{|x-y|}dy dx + 2 \int_{\Omega_i}\int_{S_i}\frac{|I_h(\overline v-{v})(x)-I_h(\overline v-{v})(y)|^2}{|x-y|}dy dx \\ \label{eq:est_dif_inter_1_weight}
    &:=2J_1+2J_2.
\end{align}
%
Using that $\overline v$ is a constant function, we can readily estimate $J_1$ as
%
\begin{equation}\label{eq:est_J1_weight}
    J_1 = \int_{\Omega_i}\int_{S_i}\frac{|v(x)-v(y)|^2}{|x-y|}dy dx \leq C\ell^{\alpha}(h)\int_{T_i}\int_{T_i}\frac{|v(x)-v(y)|^2}{|x-y|\ell^{\alpha}(|x-y|)}dy dx.
\end{equation}

On the other hand, we can apply \Cref{lem:loc_norm_interp} to $J_2$ and obtain
%
\begin{equation}\label{eq:est_J2_weight}
    J_2 \leq Ch\sum_{k\in \inter{i-1,i+2}} b_k^2
\end{equation}
%
where $b_k:=\frac{1}{h}\int_{\Omega}(v-\overline v)\varphi_k dx$. As in \eqref{abk}, for every $k\in \inter{i-1,i+2}$,
\begin{align*}
    |b_k|^2\leq \frac{C}{h}\int_{T_{i}}|v-\overline{v}|^2\, dx,
\end{align*}
because the support of $\varphi_k$ is in $T_{i}$ for all $k\in \inter{i-1,i+2}$. Using \Cref{lem:poincare_type_weight}, we can estimate the right-hand side of the above expression and obtain
%
\begin{equation}\label{eq:est_bks_weight}
    |b_k|^2\leq Ch^{-1}\ell^{\alpha}(h)\int_{T_{i}}\int_{T_{i}}\frac{|v(x)-v(y)|^2}{|x-y|\ell^{\alpha}(|x-y|)}dy dx\qquad \text{for all $k\in \inter{i-1,i+2}$}.
\end{equation}
%
Collecting estimates \eqref{eq:est_dif_inter_1_weight}--\eqref{eq:est_bks_weight} yields \eqref{eq:approxim_estimate_weight}. 

Finally, for \eqref{a4}, by \Cref{lem:poincare_type_weight}, item (ii) of \Cref{lem:loc_norm_interp}, and \eqref{eq:est_bks_weight}, 
\begin{align*}
    |v-I_h v|_{L^2(\Omega_i)}^2&\leq 2|v-\overline{v}|_{L^2(\Omega_i)}^2+2|I_h(\overline{v}-v)|_{L^2(\Omega_i)}^2\leq C\ell^{\alpha}(h)\int_{T_{i}}\int_{T_{i}}\frac{|v(x)-v(y)|^2}{|x-y|\ell^{\alpha}(|x-y|)}dy dx.
\end{align*}
\end{proof}


\begin{proposition}
Let $\alpha>0$ and $u\in\weH{1+\alpha}$, then
\begin{align}\label{eq:interp_final}
    \|u-I_hu\|^2_{\mathbb H(\Omega)}&\leq C\ell^{\alpha}(h) \|u\|^2_{\weH{1+\alpha}},
\end{align}
\end{proposition}
\begin{proof}
  By \eqref{eq:approxim_estimate_weight}, \Cref{lem:localization_enorm}, and \eqref{a4},
%
\begin{align}
    \|u-I_h u\|^2_{\mathbb H(\Omega)}&\leq C \ell^{1+\alpha}(h)\sum_{i\in\inter{0,N}}\int_{T_i}\int_{T_i}\frac{|u(x)-u(y)|^2}{|x-y|\ell^{1+\alpha}(|x-y|)}dy dx\notag\\
    &\quad +C\ell^{-1}(h)\sum_{i\in\inter{0,N}}|u-u_h|_{L^2(\Omega_i)}^2+C\sum_{i\in\{0,N\}}\int_{\Omega_i}|u(x)|^2\ell^{-1}(d(x))\,dx\notag\\
    &\leq C \ell^{\alpha}(h)\sum_{i\in\inter{0,N}}\int_{T_i}\int_{T_i}\frac{|u(x)-u(y)|^2}{|x-y|\ell^{1+\alpha}(|x-y|)}dy dx\notag\\ 
    &\quad +C\ell^{\alpha}(h)\sum_{i\in\{0,N\}}\int_{\Omega_i}|u(x)|^2\ell^{-1-\alpha}(d(x))\,dx,\label{eq:est_aprox_sum}
\end{align}
where we used that $\frac{\ell^{\alpha}(d(x))}{\ell^{\alpha}(h)}\leq 1$ for $x\in\Omega_i$.  

Let $x_i:=x_0=0$ if $i<0$ and $x_i:=x_{N+1}=L$ if $i>N+1$.  Then, 
%
\begin{align}\notag
   \sum_{i\in\inter{0,N}} \int_{T_i}\int_{T_i}\frac{|u(x)-u(y)|^2}{|x-y|\ell^{1+\alpha}(|x-y|)}dy dx 
   &\leq \sum_{i\in\inter{0,N}} \int_{x_{i-2}}^{x_{i+3}}\int_{\Omega}\frac{|u(x)-u(y)|^2}{|x-y|\ell^{1+\alpha}(|x-y|)}dy dx\notag\\
   &\leq \sum_{i\in\inter{0,N}} \sum_{j\in \inter{i-2,i+3}}\int_{\Omega_j}\int_{\Omega}\frac{|u(x)-u(y)|^2}{|x-y|\ell^{1+\alpha}(|x-y|)}dy dx\notag\\
 & \leq 5\int_{\Omega}\int_{\Omega} \frac{|u(x)-u(y)|^2}{|x-y|\ell^{1+\alpha}(|x-y|)}dy dx. \label{eq:est_aprox_omega}
\end{align}

Noting that the last integral in \eqref{eq:est_aprox_sum} can be directly extended to the set $\Omega$, we can put together estimates \eqref{eq:est_aprox_sum} and \eqref{eq:est_aprox_omega} and obtain \eqref{eq:interp_final}
\end{proof}
%


\section{Stability of the error}

We begin by recalling the variational formulation associated to our problem: find $u\in\mathbb H(\Omega)$ such that 
%
\begin{equation}\label{eq:cont_problem}
    \cE_{L}(u,v)=\int_{\Omega}f v \quad \forall v\in \mathbb H(\Omega)
\end{equation}
%
with $\cE_{L}(u,v)=\cE(u,v)+B(u,v)$ and where we recall \eqref{eq:bilinear_B}.

Our goal is to approximate the solution $u$ to \eqref{eq:cont_problem} by solving the following discrete problem: find $u_h\in \mathcal V_h$ such that
%
\begin{equation}\label{eq:discr_problem}
    \cE_{L}(u_h,v)=\int_{\Omega} f v \quad \forall v\in \mathcal V_h
\end{equation}
%


Unlike the case of the fractional Laplacian, we observe that the bilinear form $\cE_{L}(u,u)$ is not coercive in $\mathbb H(\Omega)$ so we cannot employ the classical Cea's lemma. Instead, we shall use the following variant which employs a discrete inf-sup condition. 
%
\begin{lemma}\label{lem:cea_mg}
Assume that there exists a constant $\alpha_0>0$ uniform with respect to $h$ such that
%
\begin{equation}\label{eq:discr_inf_sup}
    \sup_{v\in \mathcal V_h}\frac{\cE_{L}(u_h,v)}{\|v\|_{\mathbb H(\Omega)}}\geq \alpha_0\|u_h\|_{\mathbb H(\Omega)}, \quad\forall u_h\in\mathcal V_h.
\end{equation}
%
Then, there exists a unique solution $u_h\in \mathcal V_h$ to \eqref{eq:discr_problem}. Moreover, this solution satisfies
%
\begin{equation}\label{eq:stability_cea}
    \|u-u_h\|_{\mathbb H(\Omega)}\leq \left(1+\frac{C}{\alpha_0}\right)\inf_{w\in\mathcal V_h}\|u-w\|_{\mathbb H(\Omega)}.
\end{equation}
%
where $C>0$ is uniform with respect to $h$.
\end{lemma}
%
\begin{proof}
Let $(e_i)_i\subset \mathcal V_h$ be a finite dimensional basis of $\mathcal V_h$ with $\dim(\mathcal V_h)=K$. Setting $u_h=\sum_{i\in\inter{1,K}}u_i e_i$ and $v_h=\sum_{i\in\inter{1,K}} v_i e_i$, we note that since $\cE_{L}(\cdot,\cdot)$ is a symmetric bilinear form, the discrete problem \eqref{eq:discr_problem} amounts to solve the linear system 
%
\begin{equation}
    A_h U = F_h
\end{equation}
%
where $A_h\in\mathbb R^{K\times K}$ with entries $(A_h)_{ij}=\cE_L(e_i,e_j)$, $F\in\R^{K}$ with entries $(F)_i=\int_{\Omega}f e_i$, and $U=(u_1,\ldots, u_K)^T\in \R^{K}$. 

The discrete $\inf$-$\sup$ condition \eqref{eq:discr_inf_sup} readily ensures the non-singularity of the matrix $A_h$, thus $U$ can be uniquely determined and so $u_h$. On the other hand, since $\mathcal V_h\subset \mathbb H(\Omega)$ we can take $v\in \mathcal V_h$ in \eqref{eq:cont_problem} to obtain from \eqref{eq:discr_problem} that
%
\begin{align}\label{eq:ort_cond}
    \cE_{L}(u-u_h,v)=0 \quad\forall v\in\mathcal V_h.
\end{align}
%
Let $w\in\mathcal V_h$ be an arbitrary function in $\mathcal V_h$. By \eqref{eq:discr_inf_sup} and \eqref{eq:ort_cond} we have
%
\begin{equation}
    \alpha_0\|u_h-w\|_{\mathbb H(\Omega)} \leq \sup_{v\in\mathcal V_h}\frac{\cE_{L}(u_h-w,v)}{\|v\|_{\mathbb H}}=\sup_{v\in\mathcal V_h}\frac{\cE_{L}(u-w,v)}{\|v\|_{\mathbb H}}.
\end{equation}
%
Using \cite[Lemma 3.4]{HSS22} and the compact embedding $\mathbb H(\Omega)\hookrightarrow L^2(\Omega)$, we can see that there exists $C>0$ such that
%
\begin{equation*}
    |\cE_{L}(u-w,v)|\leq C\|u-w\|_{\mathbb H(\Omega)}\|v\|_{\mathbb H(\Omega)}
\end{equation*}
%
whence
%
\begin{align}\label{eq:est_stab_w}
    \alpha_0\|u_h-w\|_{\mathbb H(\Omega)} \leq C\|u-w\|_{\mathbb H(\Omega)} \quad\forall w\in\mathcal V_h.
\end{align}
Finally, triangle inequality and \eqref{eq:est_stab_w} yield
%
\begin{equation}
    \|u-u_h\|_{\mathbb H(\Omega)}\leq \|u-w\|_{\mathbb H(\Omega)}+\|w-u_h\|_{\mathbb H(\Omega)}\leq \left(1+\frac{C}{\alpha_0}\right)\|u-w\|_{\mathbb H(\Omega)}.
\end{equation}
%
Taking the infimum for $w\in \mathcal V_h$ yields the desired result. 
\end{proof}

\begin{lemma}
Assume that the problem $L_\Delta u=0$ in $\Omega$ with $u=0$ in $\R\backslash  \Omega$ only has the trivial solution.  Then, there are $h_0>0$ and $\alpha_0>0$ such that
\begin{equation*}
    \sup_{v\in \mathcal V_h}\frac{\cE_{L}(u,v)}{\|v\|_{\mathbb H(\Omega)}}\geq \alpha_0\|u\|_{\mathbb H(\Omega)} \quad\text{ for all } u\in\mathcal V_h \text{ and }h\in(0,h_0),
\end{equation*}
where $\alpha_0$ is independent of $h$.
\end{lemma}
\begin{proof}
We argue as in \cite[Theorem 4.2.1]{dem20}. Assume, by contradiction, that for every $n\in\N$ there are $h_n<\frac{1}{n}$ and a sequence $(u_n)_{n\in\N}\in \mathcal V_{h_n}$ such that $\|u_n\|_{\mathbb H(\Omega)}=1$ and
\begin{equation}\label{ccea}
    \sup_{v\in \mathcal V_{h_n}}\frac{\cE_{L}(u_n,v)}{\|v\|_{\mathbb H(\Omega)}}<\frac{1}{n}\qquad \text{ for all }n\in\N.
\end{equation}
Since $(u_n)$ is bounded in $\mathbb H(\Omega)$, then there is $u\in \mathbb H(\Omega)$ such that \SJ{after passing to a subsequence} $u_n \rightharpoonup u$ weakly in $\mathbb H(\Omega)$. Moreover, since $\mathbb H(\Omega)$ is compactly embedded in $L^2(\Omega)$, then $u_n\to u$ in $L^2(\Omega)$ \SJ{after passing to another subsequence}.  Then,
\begin{align}
    \|u_n-u\|^2_{\mathbb H(\Omega)}
    &=\cE(u_n,u_n-u)-\cE(u,u_n-u)\notag\\
    &=\cE_L(u_n,u_n-u)-B(u_n,u_n-u)-\cE(u,u_n-u),\label{007}
\end{align}
where $B$ is given by \eqref{eq:bilinear_B}.  In particular, $B$ is a bilinear continuous operator in $L^2(\Omega)$.  Note that $\cE(u,u_n-u)\to 0$ as $n\to \infty$ by weak convergence in $\mathbb H(\Omega)$, $B(u_n,u_n-u)\to 0$ as $n\to \infty$ by strong convergence in $L^2(\Omega)$. By Theorem \ref{density:thm}, there is a sequence $(\widetilde u_n)$ in $\mathcal V_h$ such that $\widetilde u_n\to u$ in $\mathbb H(\Omega)$ as $n\to \infty$, as a consequence,
\begin{align*}
    \cE_L(u_n,u_n-u)
    &=\cE_L(u_n,u_n-\widetilde u_n)+\cE_L(u_n,\widetilde u_n-u)\\
    &\leq \frac{\cE_L(u_n,u_n-\widetilde u_n)}{\|u_n-\widetilde u_n\|_{\mathbb H(\Omega)}}\|u_n-\widetilde u_n\|_{\mathbb H(\Omega)}+\cE_L(u_n,\widetilde u_n-u) \to 0\qquad \text{ as }n\to \infty,
\end{align*}
by \eqref{ccea} and arguing as in \eqref{007}, because $(u_n-\widetilde u_n)_{n\in\N}$ and $(u_n)$ are bounded in $\mathbb H(\Omega)$.  As a consequence, $u_n\to u$ in $\mathbb H(\Omega)$ as $n\to\infty$. Then $\|u\|_{\mathbb H(\Omega)}=1$. Let $w\in \mathbb H(\Omega)\backslash \{0\}$ and let $(w_n)\subset \mathcal V_h$ such that $w_n\to w$ in $\mathbb H(\Omega)$ as $n\to \infty$ (see Theorem \ref{density:thm}), then
\begin{align*}
    \frac{\cE_{L}(u,w)}{\|w\|_{\mathbb H(\Omega)}}=\lim_{n\to\infty}\frac{\cE_{L}(u_n,w_n)}{\|w_n\|_{\mathbb H(\Omega)}}\leq \lim_{n\to\infty}\sup_{v\in \mathcal V_h}\frac{\cE_{L}(u_n,v)}{\|v\|_{\mathbb H(\Omega)}}\leq \lim_{n\to\infty}\frac{1}{n}=0.
\end{align*}
Since $w$ is arbitrary, this implies that $u$ is a nontrivial weak solution of $L_\Delta u=0$ in $\Omega$ with $u=0$ in $\R\backslash  \Omega$, but this contradicts our assumption and the claim follows. 
\end{proof}


% \Cref{lem:cea_mg} relies heavily on the discrete $\inf$-$\sup$ condition \eqref{eq:discr_inf_sup}. In turn, this is consequence of the general result \cite[Theorem 4.2.1]{dem20} which, in our context translates into the following: there exists $h_0>0$ and $\gamma_0>0$ such that for any $h<h_0$
% %
% \begin{equation}
%     \sup_{v\in\mathcal V_h} \frac{|\cE_{L}(u_h,v)|}{\|v\|_{\mathbb{H}}}\geq \gamma_0\|u_h\|_{\mathbb H} \quad\forall u_h\in\mathcal V_h.
% \end{equation}
% %
% Note that we have to pay the price of imposing an extra smallness assumption on the discrete parameter $h$. 

\begin{proposition}\label{prop:more_regularity}
Assume that alternative $i)$ of \Cref{eq:regularity} holds for some $\alpha\in(0,1)$. Then, $u\in \mathbb H^{1+\alpha}(\Omega;\ell)$. Moreover, there is $C>0$ only depending on $\Omega$ and $\alpha$ such that $\|u\|_{H^{1+\alpha}(\Omega;\ell)}\leq C\|u\|_{\mathcal X^{\alpha}(\Omega)}$.
\end{proposition}
%
\begin{proof}
Fix $\alpha\in(0,1)$ as given by \Cref{eq:regularity}. Let $x\in \Omega$ and let $y\in\partial \Omega$ be such that $d(x)=|x-y|$. Then $u(y)=0$ and 
%
\begin{align}\label{eq:iden_fro}
    \frac{|u(x)|}{\ell^{\alpha}(d(x))}=\frac{|u(x)-u(y)|}{\ell^{\alpha}(|x-y|)}.
\end{align}
%
From the definition of the spaces $H^{1+\alpha}(\Omega;\ell)$ and $\mathcal X^\alpha(\Omega)$ and identity \eqref{eq:iden_fro}, we have
%
\begin{align}\notag
    \|u\|^2_{H^{1+\alpha}(\Omega;\ell)}&=\int_{\Omega}\int_{\Omega}\frac{|u(x)-u(y)|^2}{|x-y|\ell^{1+\alpha}(|x-y|)}\, \d{y} \d{x}+\int_{\Omega}|u(x)|^2\ell^{-1-\alpha}(d(x))\,\d{x} \\
    &\leq \|u\|^2_{\mathcal X^{\alpha}(\Omega)}\left(\int_{\Omega}\int_{\Omega}\frac{\ell^{1+\alpha}(|x-y|)}{|x-y|\ell^{2+2\alpha}(d(x,y))}\,\d{y}\d{x}+ \int_{\Omega}\ell^{-1+\alpha}(d(x))\,\d{x}\right).\label{a6}
\end{align}
%
The claim now follows from \eqref{lem:int:bds}.
\end{proof}
%

\begin{theorem}
    Assume that alternative $i)$ of \Cref{eq:regularity} holds for some $\alpha\in(0,1)$ and let $h>0$ be a sufficiently small discretization parameter. For the solution $u$ to \eqref{eq:cont_problem} and its finite element approximation $u_h$ given by \eqref{eq:discr_problem}, we have the following a priori estimate
    %
    \begin{equation}
        \|u-u_h\|_{\mathbb H(\Omega)}\leq C\ell^{\alpha}(h)\|f\|_{\mathcal Y}
    \end{equation}
    %
    for some $\alpha\in(0,1)$ and where $C>0$ is uniform with respect to $h$ 
\end{theorem}

\begin{proof}
Fix $\alpha\in(0,1)$ as given by \Cref{eq:regularity}. Combining estimate \eqref{eq:interp_final} (with this particular $\alpha$) and \eqref{eq:stability_cea} we get
%
\begin{align}
    \|u-u_h\|_{\mathbb H(\Omega)}\leq C\ell^{\alpha}(h)\|u\|_{\mathbb{H}^{1+\alpha}(\Omega;\ell)}.
\end{align}
%
for some constant $C>0$ uniform with respect to $h$. By \Cref{prop:more_regularity} and the regularity estimate \eqref{eq:regularity_classical} we obtain the desired result.
\end{proof}

% \section{FEM setting}

% Let $\Omega=(0,b)$ with $b\in(0,1)$. Given $M\in \N$, let $I:=\{1,\ldots, M\}$, $h:=\frac{b+2}{M}$, and
% \begin{align*}
% x_0:=-1,\qquad x_i:=x_0+i h\text{ for }i\in \mathbb Z.
% \end{align*}
% In particular, $x_M=b+1$.   Furthermore, for $i\in I$, let
% \begin{align*}
% \Omega_i:=(x_{i-1},x_{i}),\qquad 
% U:=B_h(0)=(-h,h),\qquad 
% D:=B_1\backslash U=(-1,1)\backslash (-h,h).
% \end{align*}

% Then 
% \begin{align*}
%     B_1(\Omega)=(-1,b+1)=\bigcup_{i\in I} \Omega_i.
% \end{align*}

% For $i\in I,$ let $\varphi_i$ denote the usual $P_0$ or $P_1$ finite element basis in dimension one. For $u\in \cH(\Omega)\cap C(\R^N)$ let $I_hu:\R^N\to\R$ be given by
% \begin{align*}
% I_h u (x):=\sum_{i\in I}a_i\varphi_i(x),\qquad a_i:=\frac{1}{2h}\int_{\Omega} u\varphi_i.
% \end{align*}

% In the following $C>0$ denotes possibly different constants independent of $i$.
% Observe that
% % \begin{align*}
% % a_j\leq |u|_{2,\Omega_i}|\varphi_i|_2 \leq C|u|_{2,\Omega_i} h
% % \end{align*}
% % and
% \begin{align*}
% \|I_h u\|^2
% &=\int_{B_1(\Omega)}\int_{B_1}\frac{|I_h u(x)-I_h u(x+y)|^2}{|y|}\, dy\, dx\\
% &=
% \sum_{i\in I}\int_{\Omega_i}\int_{U}\frac{|I_h u(x)-I_h u(x+y)|^2}{|y|}\, dy\, dx
% +\sum_{i\in I}\int_{\Omega_i}\int_{D}\frac{|I_h u(x)-I_h u(x+y)|^2}{|y|}\, dy\, dx\\
% &=:T_1+T_2.
% \end{align*}

% Let
% \begin{align*}
% I(i)
% :=\{j\in I\::\: \varphi_j(x)-\varphi_j(x+y)\neq 0 \text{ for some }(x,y)\in \Omega_i\times U\}
% =\{i-2,\ldots,i+2\}.
% \end{align*}

% Then, using the Lipschitz continuity,
% \begin{align*}
% T_{1,i}&:=\int_{\Omega_i}\int_{U}\frac{|I_h u(x)-I_h u(x+y)|^2}{|y|}\, dy\, dx
% =\int_{\Omega_i}\int_{U}\frac{|\sum_{j\in I}a_j(\varphi_j(x)-\varphi_j(x+y))|^2}{|y|}\, dy\, dx\\
% &\leq \left(\sum_{j\in I(i)}a_j^2\right)\sum_{j\in I(i)}\int_{\Omega_i}\int_{U}\frac{|\varphi_j(x)-\varphi_j(x+y)|^2}{|y|}\, dy\, dx\leq \frac{C}{h^2} \left(\sum_{j\in I(i)}a_j^2\right)\sum_{j\in I(i)}\int_{\Omega_i}\int_{U}|y|\, dy\, dx\\
% &=\frac{C}{h^2} \left(\sum_{j\in I(i)}a_j^2\right)\sum_{j\in I(i)}|\Omega_i| h^2
% \leq C h\left(\sum_{j\in I(i)}a_j^2\right).%\leq Ch^2|u|_{2,\Omega_i}.
% \end{align*}

% Then
% \begin{align*}
% T_1=\sum_{i\in I}T_{1,i}\leq Ch \sum_{i\in I}\left(\sum_{j\in I(i)}a_j^2\right).
% \end{align*}

% On the other hand,
% \begin{align*}
% T_{2,i}
% &:=\int_{\Omega_i}\int_{D}\frac{|I_h u(x)-I_h u(x+y)|^2}{|y|}\, dy\, dx\\
% &\leq 2\left(
% \int_{\Omega_i}|I_h u(x)|^2\int_{D}|y|^{-1}\, dy\, dx
% +\int_{\Omega_i}\int_{D}|I_h u(x+y)|^2|y|^{-1}\, dy\, dx
% \right).
% \end{align*}
% Note that
% \begin{align*}
%     \sum_{i\in I}\int_{\Omega_i}|I_h u(x)|^2\int_{D}|y|^{-1}\, dy\, dx
%     \leq C|\ln(h)|\int_{\Omega_i}|I_h u(x)|^2\, dx
%     %&\leq C|\ln(h)|\int_{\Omega}\left|\sum_{j\in I}a_j\varphi_j(x)\right|^2\, dx\\
%     %&\leq C|\ln(h)|\left(\sum_{j\in I}a_i^2\right)
% \end{align*}
% and
% \begin{align*}
% \int_{\Omega_i}\int_{D}|I_h u(x+y)|^2|y|^{-1}\, dy\, dx
% \leq \int_{\Omega_i}\int_{D}|I_h u(x+y)|^2|y|^{-1}\, dy\, dx
% \end{align*}

% \section{The logarithmic Hardy inequality}

% Let $\Omega\subset \R^N$ be a bounded domain.

% \begin{lemma}\label{lem:equiv_norm}
% For every $R>0$, the norm
% \begin{align*}
%     \|u\|_{R}:=\left(\int_{\R^N}\int_{B_R(x)}\frac{|u(x)-u(y)|^2}{|x-y|^N}\, dy\, dx\right)^\frac{1}{2}
% \end{align*}
% is an equivalent norm in $\mathbb H(\Omega)$.
% \end{lemma}
% \begin{proof}
% TO-DO
% \end{proof}

% Now we show a logarithmic Hardy inequality for domains satisfying a uniform exterior sphere condition, namely, that there is $r_0>$ such that for every $z\in \partial\Omega$ there is $z^*\in \R^N\backslash \Omega$ such that
% \begin{align}\label{r0}
% \overline{B_{r_0}(z^*)}\cap \overline{\Omega}=\{z\}.
% \end{align}

%  We follow the argument in \cite[Section 4.2]{BC18} devised for convex domains. For $x\in \R^N$, let
% \begin{align*}
% \delta(x)=\delta_\Omega(x):=\chi_{\Omega}(x)\inf_{y\in \partial \Omega}|x-y|.
% \end{align*}

% \begin{proposition}\label{prop:hardy}
% Let $\Omega\subset \R^N$ be a bounded domain satisfying a uniform exterior sphere condition. Then
% \begin{align*}
% \int_{\Omega}\frac{|u(x)|^2}{\ell(\delta_\Omega(x))}\, dx\leq C\|u\|^2\qquad \text{ for all }u\in\mH(\Omega)
% \end{align*}
% and for some constant $C=C(\Omega)>0.$
% \end{proposition}
% \begin{proof}
% Let $R:=2\diam(\Omega)>0$. For every $x\in \Omega$ there is $x_0\in \R^N$ such that
% \begin{align*}
%     |x-x_0|=2\delta(x),\qquad \frac{x+x_0}{2}\in \partial \Omega.
% \end{align*}
% For every $x\in\Omega$ and $y\in \R^N\backslash B_{\delta(x)}(x_0)$,
% \begin{align*}
% |x-y|\leq |x-x_0|+|x_0-y|=2\delta(x)+|x_0-y|\leq 3|x_0-y|.
% \end{align*}
% We now split the domain $\Omega.$ Let $r_0>0$ be as in \eqref{r0} and let
% \begin{align*}
%     \Omega_1:=\left\{y\in\Omega\::\: \delta(y)<\frac{r_0}{10}\right\},\qquad \Omega_2:=\Omega\backslash \Omega_1.
% \end{align*}
% By making $r_0$ smaller if necessary, we can assume that, for every $x\in\Omega_1$,
% \begin{align*}
% A_x:=\{y\in \R^N\::\: \delta(x)<|y-x_0|<r_0,\ (y-x_0,x-x_0)<0\}\subset B_R(x)\backslash \Omega\cup B_{\delta(x)}(x_0)).
% \end{align*}

% \setlength{\unitlength}{1cm}
% \begin{figure}[ht] 
% \begin{center}
% \begin{picture}(5,5)
% \put(0,0){\includegraphics[width=5cm]{v.png}}
% \put(0.5,4){$\Omega$}
% \put(3.5,2){$A_x$}
% \put(2.77,2.54){$\bullet$}
% \put(2.7,2.3){$x_0$}
% \put(2.17,2.54){$\bullet$}
% \put(2.1,2.3){$x$}
% \end{picture}
% \end{center}
%  \caption{The set $A_x$.}\label{f}
% \end{figure}




% Then
% \begin{align}
%     \int_{\R^N}\int_{B_R(x)}\frac{|u(x)-u(y)|^2}{|x-y|^N}\, dy\, dx
%     &\geq \int_{\Omega}|u(x)|^2\int_{B_R(x)\backslash (\Omega\cup B_{\delta(x)}(x_0))}|x-y|^{-N}\, dy\, dx\notag\\
%     =T_1+T_2,\label{H1}
% \end{align}
% where
% \begin{align}
%     T_1&:=\int_{\Omega_1}|u(x)|^2\int_{B_R(x)\backslash (\Omega\cup B_{\delta(x)}(x_0))}|x-y|^{-N}\, dy\, dx \notag\\
%     &\geq 3^{-N}\int_{\Omega_1}|u(x)|^2\int_{A_x}|x_0-y|^{-N}\, dy\, dx\geq \frac{1}{2\omega_N 3^N}\int_{\Omega_1}|u(x)|^2\int_{\delta(x)}^{r_0}\rho^{-1}\, d\rho\, dx\notag\\
%     &=\frac{1}{2\omega_N 3^N}\int_{\Omega_1}|u(x)|^2\left|\ln\left(\frac{\delta(x)}{r_0}\right)\right|\, dx\geq C_1\int_{\Omega_1}\frac{|u(x)|^2}{\ell(\delta(x))}\, dx\label{H2}
% \end{align}
% for some $C_1=C_1(\Omega)>0$ (using Lemma \ref{prop1} together with the fact that $\frac{\delta(x)}{r_0}<\frac{1}{10}$) and
% \begin{align}
%     T_2&:=\int_{\Omega_2}|u(x)|^2\int_{B_R(x)\backslash (\Omega\cup B_{\delta(x)}(x_0))}|x-y|^{-N}\, dy\, dx\notag\\
%     &\geq (3R)^{-N}|B_R| \int_{\Omega_2}|u(x)|^2\frac{\ell(\delta(x))}{\ell(\delta(x))}\, dx
%     \geq C_2\int_{\Omega_2}\frac{|u(x)|^2}{\ell(\delta(x))}\, dx\label{H3}
% \end{align}
% for some $C_2=C_2(\Omega)>0$, because $|x-y|<3R$ for $x\in \Omega$ and $y\in B_R(x)$ and using that $\ell(\delta(x))\geq \ell(\frac{r_0}{10})$ for $x\in \Omega_2.$ The claim now follows from \eqref{H1}, \eqref{H2}, and \eqref{H3}.
% \end{proof}

% \begin{corollary}
% Let $\Omega\subset \R^N$ be a bounded domain satisfying a uniform exterior sphere condition.  There is $C=C(\Omega)>0$ such that
% \begin{align*}
%     \|u\|^2\leq C\int_{\Omega}\int_{\Omega}\frac{|u(x)-u(y)|^2}{|x-y|^N}\, dx\, dy\qquad \text{ for all }u\in\mH(\Omega).
% \end{align*}
% \end{corollary}
% \begin{proof}
% Let $u\in C_c^\infty(\Omega)$ and $R>0$ be fixed such that $\Omega\subset B_R(x)$ for all $x\in\Omega$. By \Cref{lem:equiv_norm}, we have
% %
% \begin{align}\label{eq:des_init}
%     \|u\|^2&\leq \|u\|_R^2=C\int_{\R^N}\int_{B_R(x)}\frac{|u(x)-u(y)|^2}{|x-y|^N}dy dx. 
% \end{align}
% %
% for some $C>0$ only depending on $\Omega$ and $R$. Let
% $
%     B_R(\Omega):=\{x\in \R^N\::\: \dist(x,\Omega)<R\}.
% $
% %
% {Note that if $\dist(x,\Omega)>R$ and $|x-y|<R$, then $y\not\in \Omega$}, whence
% %
% \begin{align}\label{eq:norm_R_bound}
%      \|u\|_R^2=C\int_{B_{R}(\Omega)}\int_{B_R(x)}\frac{|u(x)-u(y)|^2}{|x-y|^N}dy dx \leq  C\int_{B_{R}(\Omega)}\int_{B_{R^\prime}}\frac{|u(x)-u(y)|^2}{|x-y|^N}dy dx . 
% \end{align}
% %
% where we have chosen $R^\prime>0$ large enough such that $ B_R(x)\subset B_{R^\prime}$ for all $x\in B_R(\Omega)$. In particular, $B_R(\Omega)\subset B_{R^\prime}$ and hence
% %
% \begin{align}\label{eq:iden_norms}
%   \|u\|_R^2 
%     &\leq C\left(\int_{\Omega}\int_{\Omega}\frac{|u(x)-u(y)|^2}{|x-y|^N}dy dx +2\int_{\Omega}\int_{B_{R^\prime}\setminus \Omega} \frac{|u(x)|^2}{|x-y|^N}dy dx\right) 
% \end{align}
% %
% since $u=0$ in $\R^N\setminus \Omega$. 
% For the second term in the r.h.s of \eqref{eq:iden_norms}, we have that
% %
% \begin{align}\notag
%     \int_{\Omega}\int_{B_{R_2}\setminus \Omega} \frac{|u(x)|^2}{|x-y|^N}dy dx&\leq \int_{\Omega}|u(x)|^2\int_{B_{R^\prime}\setminus B_{\delta(x)}(x)}\frac{1}{|x-y|^N}dy dx \\ \label{eq:final_bound}
%     &=\int_{\Omega}|u(x)|^2\log\left(\frac{R^\prime}{\delta(x)}\right)dx \leq C \int_{\Omega}\frac{|u(x)|^2}{\ell(\delta(x))}dx
% \end{align}
% %
% for some constant $C>0$ depending on $R^\prime$.   Putting together \eqref{eq:des_init}, \eqref{eq:norm_R_bound}, \eqref{eq:iden_norms}, and \eqref{eq:final_bound} and using \Cref{prop:hardy} yield that
% %
% \begin{align}
%     \|u\|^2\leq C \int_{\Omega}\int_{\Omega}\frac{|u(x)-u(y)|^2}{|x-y|^N}dy dx + C\|u\|^2
% \end{align}
% %
% for some $C>0$ only depending on $\Omega$, $R$ and $R^\prime$. 
% %
% \end{proof}





\section{The stiffness matrix for the logarihtmic Laplacian}

Recall that $\Omega:=(0,L)$ for some $L>0$.  We use $A^L_h$ to denote the stiffness matrix associated to the logarithmic Laplacian $L_\Delta.$   The entries of the matrix $A^L_h=(b_{ij})_{i,j=0}^{N+1}$ can be obtained by computing
\begin{align*}
b_{ij}=\cE_L(\varphi_i,\varphi_j)\qquad \text{ for }i,j=0,\ldots,N+1.
\end{align*}
Instead of doing this computation directly, we obtain a closed formula for $b_{ij}$ by differentiating the entries of the stiffness matrix for the fractional Laplacian (which were explicitly computed in \cite{BH17}) and evaluating them at $s=0$.  To be more precise, let $A^s_h=(b_{ij}^s)_{i,j=0}^{N+1}$ be the stiffness matrix for $(-\Delta)^s$. We show the following. 
\begin{lemma}\label{lem:derivative:s:m} For $i,j=0,\ldots,N+1,$
\begin{align*}
    b_{ij}=\partial_s b_{ij}^s \mid_{s=0}.
\end{align*}
\end{lemma}
\begin{proof}
We claim that 
\begin{align}\label{aux:1}
\lim_{s\to 0}\left\|\frac{(-\Delta)^s \varphi_i-\varphi_i}{s}-L_\Delta \varphi_i    \right\|_{L^p(\R)}=0\qquad \text{for $i=0,\ldots,N+1$ and $1<p<\infty$.}
\end{align}
For $i=1,\ldots,N$, this follows directly from \cite[Theorem 1.1]{CW19}, because $\varphi_i$ is Lipschitz continuous in $\R$.  However, the functions $\phi_0$ and $\phi_{N+1}$ have a jump discontinuity at 0 and at $L$, respectively.  Nevertheless, the proof of \cite[Theorem 1.1]{CW19} can be adapted to this case with obvious changes.  For the reader's convenience we include a brief proof in the appendix (see Lemma \ref{phi0:conv}). Then, integrating by parts (see Lemma \ref{ibyp:lem}) and using \eqref{aux:1},
\begin{align*}
\partial_s b_{ij}^s\mid_{s=0}&=\partial_s\cE_s(\varphi_i,\varphi_j)\mid_{s=0}=
\partial_s\int_{\R} (-\Delta)^s\varphi_i \varphi_j\, dx\mid_{s=0}\\
&=\int_{\R} \partial_s(-\Delta)^s\varphi_i\mid_{s=0} \varphi_j\, dx
=\int_{\R} L_\Delta\varphi_i \varphi_j\, dx
=\cE_L(\varphi_i,\varphi_j)=b_{ij},
\end{align*}
where we used that
\begin{align*}
\int_{\R}\left(\frac{(-\Delta)^s \varphi_i-\varphi_i}{s}-L_\Delta \varphi_i  \right)\varphi_j\, dx
\leq \|\varphi_j\|_\infty\int_{\Omega}\left|\frac{(-\Delta)^s \varphi_i-\varphi_i}{s}-L_\Delta \varphi_i  \right|\, dx\to 0
\end{align*}
as $s\to 0^+,$ since $\Omega$ is bounded.
\end{proof}


In \cite{L22}, a good numerical approximation of these coefficients is done using a trapeze formula.  The error involved in this approximation is of the order $O(\frac{1}{N^2})$.  








\section{Computation of the stiffness matrix}\label{sec:s:m}

In this section we construct the stiffness matrix associated to the fractional Laplacian $(-\Delta)^s$ with $s\in(0,\frac{1}{2})$. We follow the computations done in \cite{BH17}.  In particular, we want to calculate
\begin{align*}
a_{ij}:=\frac{2}{c_{1,s}}\cE_s(\varphi_i,\varphi_j)\qquad \text{ for }i,j=0,\ldots, N+1,
\end{align*}
where $c_{1,s}=4^s\pi^{-1/2}s(1-s)\frac{\Gamma(1/2+s)}{\Gamma(2-s)}=\frac{\sin (\pi  s) \Gamma (2 s+1)}{\pi }$.

If $i,j\neq 0,N+1$, then these coefficients were obtained in \cite{BH17}, and are given by the following formulas. 

If $i,j\in\inter{1,N}$ and $j-i\geq 2$, then
\begin{align*}
	a_{i,j} = - h^{1-2s}\,\frac{4(k+1)^{3-2s} + 4(k-1)^{3-2s}-6k^{3-2s}-(k+2)^{3-2s}-(k-2)^{3-2s}}{2s(1-2s)(1-s)(3-2s)};
\end{align*} 
moreover, for $i\in\inter{1,N-1}$,
\begin{align*}
	a_{i,i+1} =	\displaystyle h^{1-2s}\frac{3^{3-2s}-2^{5-2s}+7}{2s(1-2s)(1-s)(3-2s)},
\end{align*}
and, for  $i\in\inter{1,N}$,
\begin{align*}
	a_{i,i} =
			\displaystyle h^{1-2s}\,\frac{2^{3-2s}-4}{s(1-2s)(1-s)(3-2s)}.
\end{align*}

Let $\mathcal A^s_h=\frac{c_{1,s}}{2}(a_{ij})_{i,j=0}^{N+1}$. It remains to calculate $a_{0,i}$ for $i\in\inter{0,N+1}$.  By symmetry, we have that $a_{0,0}=a_{N+1,N+1}$, $a_{0,i}=a_{i,0}=a_{N+1-i,N+1}=a_{N+1,N+1-i}$ for $i\in\inter{1,N}$. Therefore, we only consider the following cases.

\subsection{Case 1: $a_{0,0}$.}

Note that
	\begin{align*}
	a_{0,0}= & \int_{\RR}\int_{\RR}\frac{(\varphi_0(x)-\varphi_0(y))^2}{|x-y|^{1+2s}}\,dxdy
	\\
	=& 2\int_{h}^{+\infty}\int_{0}^{h} \ldots\,dxdy  + \int_{0}^{h}\int_{0}^{h} \ldots\,dxdy + 2\int_{-\infty}^{0}\int_{0}^{h} \ldots\,dxdy,
\end{align*}
where 

% In Fig. \ref{upp_dia}, we give a scheme of the regions of interactions between the basis functions $\varphi_i(x)$ and $\varphi_i(y)$ enlightening the domain of integration of the $R_i$. The regions in grey are the ones that produce a contribution to $a_{0,0}$, while on the regions in white the integrals will be zero.
% \begin{figure}
% \figinit{0.7pt}
% % Axes
% \figpt 0:(-50,70)
% \figpt 1:(-100,80) \figpt 2:(220,80)
% \figpt 3:(-40,-20) \figpt 4:(-40,290)

% %%%%
% \figpt 5:(40,0) \figpt 6:(40,270)
% \figpt 7:(65,0) \figpt 8:(65,270)
% \figpt 9:(90,0) \figpt 10:(90,270)
% %
% \figpt 11:(-80,160) \figpt 12:(200,160)
% \figpt 13:(-80,185) \figpt 14:(200,185)
% \figpt 15:(-80,210) \figpt 16:(200,210)

% \figpt 53:(65,250) \figpt 54:(65,250)
% \figpt 55:(65,230) \figpt 56:(65,25)
% \figpt 57:(65,41) \figpt 58:(200,210)

% \figpt 59:(-25,185) \figpt 60:(140,185)
% \figpt 61:(-5,185) \figpt 62:(160,185)
% \figpt 63:(65,230) \figpt 64:(65,25)

% %

% % Grey part 
% \figpt 29:(40,210) \figpt 30:(90,210)
% \figpt 31:(40,160) \figpt 32:(90,160)

% % Diagonal 
% \figpt 33:(-40,80) \figpt 34:(150,270)

% % Points for writing
% \figpt 35:(-49,164) \figpt 36:(-47,189) \figpt 37:(-49,215) 
% \figpt 38:(30,74) \figpt 39:(58,74) \figpt 40:(80,74) 

% \figpt 41:(150,240) \figpt 42:(65,240) \figpt 43:(150,185) 
% \figpt 44:(-15,240) \figpt 45:(52.5,172.5) \figpt 46:(77.5,172.5) 
% \figpt 47:(52.5,195.5) \figpt 48:(77.5,195.5) \figpt 49:(-15,185) 
% \figpt 50:(65,30) \figpt 51:(150,30) \figpt 52:(-15,30) 

% \figpt 73:(-52,280) \figpt 74:(210,70)


% % 2. Creation of the graphical file
% \figdrawbegin{}

% \figset(fillmode=yes, color=0.9)
% \figdrawline[29,30,10,6,29]
% \figdrawline[30,32,12,16,30]
% \figdrawline[32,31,5,9,32]
% \figdrawline[31,11,15,29,31]
% \figset(fillmode=yes, color=0.6)
% \figdrawline[29,30,32,31,29]
% \figset(fillmode=no, color=0)
% \figdrawarrow[1,2]
% \figdrawarrow[3,4]
% \figdrawline[5,6]
% \figdrawline[9,10]
% \figdrawline[11,12]
% \figdrawline[15,16]
% \figset(dash=4)
% \figdrawline[7,56]
% \figdrawline[57,55]
% \figdrawline[53,8]
% \figdrawline[13,59]
% \figdrawline[60,61]
% \figdrawline[62,14]
% \figset(dash=5)
% \figdrawline[33,34]
% \figdrawend

% % 3. Writing text on the figure
% \figvisu{\figBoxA}{}{
% \figwritec [0]{$O$}
% \figwritec [35,38]{$x_{i-1}$}
% \figwritec [36,39]{$x_i$}
% \figwritec [37,40]{$x_{i+1}$}
% \figwritec [73]{$y$}
% \figwritec [41]{$R_1$}
% \figwritec [42,43]{$R_2$}
% \figwritec [44]{$R_3$}
% \figwritec [45]{$R_4^1$}
% \figwritec [46]{$R_4^2$}
% \figwritec [47]{$R_4^3$}
% \figwritec [48]{$R_4^4$}
% \figwritec [49,50]{$R_5$}
% \figwritec [51]{$R_6$}
% \figwritec [52]{$R_7$}
% }
% \centerline{\box\figBoxA}
% \caption{Interactions between the basis function $\varphi_i(x)$ and $\varphi_i(y)$.}\label{dia}
% \end{figure}
% %\begin{figure}
% %	\centering
% %	\includegraphics[scale=1]{figure16.eps}
% %	\caption{Interactions between the basis function $\varphi_i(x)$ and $\varphi_i(y)$.}\label{dia}
% %\end{figure}

% Le us now compute the terms $R_i$, $i=1,\ldots,7$, separately. First, note that $R_1=R_3=R_6=R_7=0,$ because the integrating regions are outside the support of $\varphi_0$. 
% \subsubsection*{Computation of $R_2$}
\begin{align*}
	R_1 &:= 2\int_{h}^{+\infty}\int_{0}^{h} \frac{\varphi_0^2(x)}{|x-y|^{1+2s}}\,dxdy = \B{4}\int_{0}^{h}\left(1-\frac{x}{h}\right)^2\left(\int_{h}^{+\infty} \frac{1}{(y-x)^{1+2s}}\, dy\right)\,dx \\
	&= \frac{\B{2}}{s}\int_{0}^{h}\frac{\left(1-\frac{x}{h}\right)^2}{(h-x)^{2s}}\,dx
	=\frac{\B{2}h^{1-2 s}}{s(3-2 s)}.
\end{align*}

%\subsubsection*{Computation of $R_4$}
\begin{align*}
	R_2 &
	:= \int_{0}^{h}\int_{0}^{h} \frac{(\varphi_0(x)-\varphi_0(y))^2}{|x-y|^{1+2s}}\,dxdy
	= \B{2}h^{-2}\int_{0}^{h}\int_{0}^{h} |x-y|^{1-2s}\,dxdy
	=\frac{\B{2}h^{1-2 s}}{(1-s) (3-2 s)}
\end{align*}



%\subsubsection*{Computation of $R_5$}
\begin{align*}
	R_3 &:= 2\int_{-\infty}^{0}\int_{0}^{h} \frac{\varphi_0^2(x)}{|x-y|^{1+2s}}\,dxdy = 2\int_{0}^{h}\varphi_0^2(x)\left(\int_{-\infty}^{0} \frac{dy}{|x-y|^{1+2s}}\right)\,dx = \frac{1}{s}\int_{0}^{h}\frac{\varphi_0^2(x)}{x^{2s}}\,dx\\
	&=\frac{\B{2}h^{1-2 s}}{(1-s)(3-2 s) (1-2 s)s}.
\end{align*}
%
as long as $s<1/2$.

Therefore,
%
\begin{equation*}
    a_{0,0}=\frac{\B{4}h^{1 - 2 s}}{s (3 - 2 s) (1 - 2 s)}
\end{equation*}
%

\subsection{Case 2: $a_{0,j}$ for $j\in\inter{2,N}$.}

\begin{align*}
	a_{0,j}&=-2 \int_{x_{j-1}}^{x_{j+1}}\int_{0}^{h}\frac{\varphi_0(x)\varphi_j(y)}{|x-y|^{1+2s}}\,dxdy
=-\B{2\sqrt{2}} \int_{(j-1)h}^{(j+1)h}\int_0^{h}\frac{\left(1-\frac{x}{h}\right)\left(1-\frac{|y-jh|}{h}\right)}{|x-y|^{1+2s}}\,dxdy.
\end{align*}

Let us introduce the following change of variables:
\begin{align*}
	\frac{x}{h}=\hat{x},\;\;\; \frac{y-jh}{h}=\hat{y}.
\end{align*}

Then, rewriting (with some abuse of notations since there is no possibility of confusion) $\hat{x}=x$ and $\hat{y}=y$, we get 
\begin{align}\label{elem_noint_cv}
	a_{0,j}&=-\B{2\sqrt{2}}h^{1-2s} \int_{-1}^1\int_{0}^1\frac{(1-x)(1-|y|)}{(j+y-x)^{1+2s}}\,dxdy\\
	&=-\B{2\sqrt{2}}h^{1-2s} \left(\int_{0}^1\int_{0}^1\frac{(1-x)(1-y)}{(j+y-x)^{1+2s}}\,dxdy
	+\int_{-1}^0\int_{0}^1\frac{(1-x)(1+y)}{(j+y-x)^{1+2s}}\,dxdy\right)\\
	&=-\B{2\sqrt{2}}h^{1-2s}(B_1+B_3).
\end{align}
Integrating by parts several times, we have
\begin{align*}
	& B_1 = \frac{1}{4s(1-2s)}\left[2j^{1-2s}-\frac{(j+1)^{2-2s}-(j-1)^{2-2s}}{1-s}-\frac{2j^{3-2s}-(j+1)^{3-2s}-(j-1)^{3-2s}}{(1-s)(3-2s)}\right],\\
%	& B_2 = \frac{1}{4s(1-2s)}\left[-2k^{1-2s}+\frac{2(k+1)^{2-2s}-2k^{2-2s}}{1-s}+\frac{2(k+1)^{3-2s}-k^{3-2s}-(k+2)^{3-2s}}{(1-s)(3-2s)}\right]\\
	& B_3 = \frac{1}{4s(1-2s)}\left[-2j^{1-2s}+\frac{2j^{2-2s}-2(j-1)^{2-2s}}{1-s}+\frac{2(j-1)^{3-2s}-j^{3-2s}-(j-2)^{3-2s}}{(1-s)(3-2s)}\right].
%	& B_4 = \frac{1}{4s(1-2s)}\left[2k^{1-2s}-\frac{(k+1)^{2-2s}-(k-1)^{2-2s}}{1-s}-\frac{2k^{3-2s}-(k+1)^{3-2s}-(k-1)^{3-2s}}{(1-s)(3-2s)}\right].
\end{align*} 

Therefore,
\begin{align*}
	a_{0,j} =-\frac{h^{1-2 s} \gamma_j^s }{\B{\sqrt{2}}(1-2 s) (1-s) s (3- 2s )}.
\end{align*} 
where
%
\begin{align}\notag
\gamma_j^s= &-3 j^{3-2 s}+2 (3-2 s) j^{2-2 s}+(2 s-3)
   (j-1)^{2-2 s}+3 (j-1)^{3-2 s} \\ \label{eq:def_gamma_j}
   &-(j-2)^{3-2 s}+(j+1)^{3-2 s}+(2
   s-3) (j+1)^{2-2 s}.
\end{align}
%

\subsection{Case 3: $a_{0,N+1}$.} Since $\Omega=(0,L)$, we have that $x_{N+1}=L=(N+1)h=x_{N+2}$ and $x_{N-1}=L-h=h(N-1)$. Then

\begin{align*}
a_{0,N+1}&=-2 \int_{x_{N}}^{x_{N+1}}\int_{x_0}^{x_1}\frac{\varphi_0(x)\varphi_{N+1}(y)}{|x-y|^{1+2s}}\,dxdy
=-\B{4}\int_{Nh}^{(N+1)h}\int_0^{h}\frac{\left(1-\frac{x}{h}\right)\left(\frac{y}{h}-N\right)}{(y-x)^{1+2s}}\,dxdy\\
&=-\B{4}h^{1-2s}\int_{0}^{1}\int_0^{1}\frac{\left(1-x\right)y}{(y+N-x)^{1+2s}}\,dxdy,
\end{align*}
then
%
\begin{align*}
a_{0,N+1} = -\frac{h^{1-2 s} \zeta_N^s }{s(1-s)(1-2s )(3-2 s)}
\end{align*}
%
where 
%
\begin{align}\notag 
\zeta_N^s:= (N-1)^{-2 s} N^{-2 s} (N+1)^{-2 s}
   &\left[(N+1)^{2 s} \left(2 N^2 (N-1)^{2 s} (N+2 s-3)-(N-1)^3
   N^{2 s}\right) \right. \\  & \label{eq:def_zeta_N} \;
   \left. -(N+1) (N-1)^{2 s} N^{2 s} \left(N^2+4 N (s-1)+4
   s^2-6 s+1\right)\right]
\end{align}
%


\subsection{Case 4: $a_{0,1}$.}

We have 
	\begin{align*}
	a_{0,1}= & \int_{\RR}\int_{\RR}\frac{(\varphi_0(x)-\varphi_0(y))(\varphi_{1}(x)-\varphi_{1}(y))}{|x-y|^{1+2s}}\,dxdy
	\\
	= & \int_{h}^{+\infty}\int_{h}^{+\infty} \ldots\,dxdy + 2\int_{h}^{+\infty}\int_{0}^{h} \ldots\,dxdy + 2\int_{h}^{+\infty}\int_{-\infty}^{0} \ldots\,dxdy 
	\\
	& + \int_{0}^{h}\int_{0}^{h} \ldots\,dxdy + 2\int_{0}^{h}\int_{-\infty}^{0} \ldots\,dxdy + \int_{-\infty}^{0}\int_{-\infty}^{0} \ldots\,dxdy 
	\\
	:= & Q_1 + Q_2 + Q_3 + Q_4 + Q_5 + Q_6.
\end{align*}

Le us now compute the terms $Q_i$, $i=1,\ldots,6$, separately.   Note that $Q_6=Q_3=Q_1 = 0.$ Moreover, by Fubini's theorem,
\begin{align*}
	Q_2 &= 2\int_{h}^{+\infty}\int_{0}^{h} \frac{\varphi_0(x)(\varphi_{1}(x)-\varphi_{1}(y))}{|x-y|^{1+2s}}\,dxdy\\
	&= 2\int_{0}^{h}\varphi_0(x)\varphi_{1}(x)\left(\int_{h}^{+\infty} \frac{dy}{|x-y|^{1+2s}}\right)\,dx - 2\int_{h}^{2h}\int_{0}^{h} \frac{\varphi_0(x)\varphi_{1}(y)}{|x-y|^{1+2s}}\,dxdy 
	\\
	&= \frac{1}{s}\int_{0}^{h}\frac{\varphi_0(x)\varphi_{1}(x)}{(h-x)^{2s}}\,dx - 2\int_{h}^{2h}\int_{0}^{h} \frac{\varphi_0(x)\varphi_{1}(y)}{|x-y|^{1+2s}}\,dxdy
	\\
	&= \frac{\B{\sqrt{2}}}{s}\int_{0}^{h}\frac{\left(1-\frac{x}{h}\right)\left(1-\frac{h-x}{h}\right)}{(h-x)^{2s}}\,dx - \B{2\sqrt{2}}\int_{h}^{2h}\int_{0}^{h} \frac{\left(1-\frac{x}{h}\right)\left(1-\frac{y-h}{h}\right)}{|x-y|^{1+2s}}\,dxdy=:Q_2^1+Q_2^2,
\end{align*}
where, using a change of variables,
\begin{align*}
Q_2^1&=\B{\sqrt{2}}\frac{h^{1-2s}}{s}\int_0^1 x^{1-2s}(1-x)\,dx = \frac{h^{1-2s}}{s(1-s)(3-2s)\B{\sqrt{2}}},\\
Q_2^2 &= -\B{2\sqrt{2}}h^{1-2s}\int_0^1\int_0^1\frac{(1-x)(1-y)}{(y-x+1)^{1+2s}}\,dxdy = h^{1-2s}\frac{2^{2-2s}+2s-4}{s(1-s)(3-2s)\B{\sqrt{2}}}.
\end{align*}
Adding the two contributions, we get the following expression for the term $Q_2$
\begin{align*}
	Q_2 = h^{1-2s}\frac{2^{2-2s}+2s-3}{s(1-s)(3-2s)\B{\sqrt{2}}}.
\end{align*}
Finally, 
\begin{align*}
	Q_4 &= \int_{0}^{h}\int_{0}^{h} \frac{(\varphi_0(x)-\varphi_0(y))(\varphi_{1}(x)-\varphi_{1}(y))}{|x-y|^{1+2s}}\,dxdy. 
\end{align*}
Note that
\begin{align*}
	(\varphi_0(x)-\varphi_0(y))(\varphi_{1}(x)-\varphi_{1}(y)) = \B{\sqrt{2}}\left(\frac{y-x}{h}\right)\left(\frac{x-y}{h}\right) = -\B{\sqrt{2}}\frac{|x-y|^2}{h^2},
\end{align*}
and the integral becomes
\begin{align*}
	Q_4 &= -\B{\frac{\sqrt{2}}{h^2}}\int_{0}^{h}\int_{0}^{h} |x-y|^{1-2s}\,dxdy = -\B{\sqrt{2}}\frac{h^{1-2s}}{(1-s)(3-2s)}.
\end{align*}

By Fubini's theorem,
\begin{align*}
	Q_5 &= 2\int_{0}^{h}\varphi_0(y)\varphi_{1}(y)\left(\int_{-\infty}^{0} \frac{dx}{|x-y|^{1+2s}}\right)dy = \frac{1}{s}\int_{0}^{h}\frac{\varphi_0(y)\varphi_{1}(y)}{(y)^{2s}}\,dy \\
	&= \frac{\B{\sqrt{2}}}{s}\int_{0}^{h}\frac{\left(1-\frac{y}{h}\right)\left(1-\frac{h-y}{h}\right)}{y^{2s}}\,dx=\frac{h^{1-2 s}}{(1-s) s (3-2 s)\B{\sqrt{2}}}.
\end{align*}

Then,
\begin{align*}
a_{0,1}=Q_2+Q_4+Q_5=\frac{\left(2^{2-2 s}-2\right) h^{1-2 s}}{(3-2 s) (1-s) s\B{\sqrt{2}}}.
\end{align*}

\subsection{Conclusion}

The stiffness matrix $\mathcal A_h^s\in\mathbb R^{(N+2)\times (N+2)}$ has components
%
\begin{equation}\label{eq:def_aijs_s}
a_{i,j}=\frac{c_{1,s}}{2}\int_{\R}\int_{\R}\frac{(\phi_i(x)-\phi_i(y)(\phi_j(x)-\phi_j(y))}{|x-y|^{1+2s}}dx dy, \quad i,j\in\inter{0,1}, 
\end{equation}
Recall that $\frac{c_{1,s}}{2}=\frac{\sin (\pi  s) \Gamma (2 s+1)}{2\pi}$.


%
In view of the symmetry of \eqref{eq:def_aijs_s} and using the basis $(\varphi_i)_{i\in\inter{0,N+1}}$ of shape funcitons \eqref{eq:def_basis}--\eqref{eq:def_basis_ext}, the coefficients $a_{i,j}$ with $i,j\in\inter{0,N+1}$ and $j\geq i$ are given by
%
% \begin{equation}
% a_{i,j}= h^{1 - 2 s}
% \begin{cases}
% \frac{8 }{s (3 - 2 s) (1 - 2 s)} & \textnormal{$i=0$, $j=0$,} \\ 
% \frac{\left(2^{2-2 s}-2\right)}{(3-2 s) (1-s) s} & \textnormal{$i=0$, $j=1$,} \\
% -\frac{\gamma_j^s }{s(1-2 s) (1-s)(3-2s)} & \parbox[t]{.3\columnwidth}{$i=0$, $j\in\inter{2,N}$, where $\gamma_j^s$ is defined in \eqref{eq:def_gamma_j},}\\
% -\frac{\zeta^s_{N} }{s(1-2 s) (1-s) (3-2s)} & \parbox[t]{.3\columnwidth}{$i=0$, $j=N+1$, where $\zeta_N^s$ is defined in \eqref{eq:def_zeta_N},}\\
% -\frac{4(k+1)^{3-2s} + 4(k-1)^{3-2s}-6k^{3-2s}-(k+2)^{3-2s}-(k-2)^{3-2s}}{2s(1-2s)(1-s)(3-2s)} & \textnormal{ $i,j\in\inter{1,N}$ and $j-i\geq 2$,} \\
% \frac{3^{3-2s}-2^{5-2s}+7}{2s(1-2s)(1-s)(3-2s)} & \textnormal{$i\in\inter{1,N-1}$ and $j=i+1$,} \\
% \frac{2^{3-2s}-4}{s(1-2s)(1-s)(3-2s)} & \textnormal{$i\in\inter{1,N}$ and $j=i$.}
% \end{cases}
% \end{equation}
%
\begin{equation}\label{stiff:frac}
a_{i,j}= \frac{h^{1 - 2 s}}{2s(1-2s)(1-s)(3-2s)}
\begin{cases}
\B{8}(1-s) & \textnormal{$i=0$, $j=0$,} \\
\B{\sqrt{2}}\left(2^{2-2 s}-2\right)(1-2s) & \textnormal{$i=0$, $j=1$,} \\
-\B{\sqrt{2}}\gamma_j^s  & \parbox[t]{.3\columnwidth}{$i=0$, $j\in\inter{2,N}$, where $\gamma_j^s$ is defined in \eqref{eq:def_gamma_j},}\\
-\B{2}\zeta^s_{N}  & \parbox[t]{.3\columnwidth}{$i=0$, $j=N+1$, where $\zeta_N^s$ is defined in \eqref{eq:def_zeta_N},}\\
-\xi_k^s & \textnormal{ $i,j\in\inter{1,N}$ with $k\geq 2$,} \\
3^{3-2s}-2^{5-2s}+7 & \textnormal{$i\in\inter{1,N-1}$ and $j=i+1$,} \\
4^{2-2s}-8 & \textnormal{$i\in\inter{1,N}$ and $j=i$,}
\end{cases}
\end{equation}
where $k:=j-i$ and $\xi^s_k:=4(k+1)^{3-2s} + 4(k-1)^{3-2s}-6k^{3-2s}-(k+2)^{3-2s}-(k-2)^{3-2s}$


\section{Stiffness matrix of the logarithmic Laplacian}\label{sec:s:m:log}

The stiffness matrix $\mathcal A_{h}^{L}\in{\R^{(N+2)\times(N+2)}}$ of the logarithmic Laplacian has coefficients $a_{i,j}$ with $i,j\in\inter{0,N+1}$ and $j\geq 1$ given by
%
\begin{equation}\label{stiff:LogLap}
a_{i,j}= h
\begin{cases}
\displaystyle -\frac{\B{2} \gamma }{3}+\frac{\B{16}}{9}-\frac{\B{4}}{3}  \log (h)+\frac{\B{4}}{3}  \log
   (2)+\frac{\B{2}}{3}  \psi \left(\frac{1}{2}\right), & \textnormal{$i=0$, $j=0$,} \\
\displaystyle -\frac{\gamma }{3\B{\sqrt{2}}}+\frac{5}{9\B{\sqrt{2}}}-\frac{\B{\sqrt{2}}}{3} \log (h)-\frac{\B{\sqrt{2}}}{3}  \log
   (2)+\frac{1}{3\B{\sqrt{2}}} \psi \left(\frac{1}{2}\right), & \textnormal{$i=0$, $j=1$,} \\
   -\frac{\B{\sqrt {2}}}{3},
 & \parbox[t]{.3\columnwidth}{$i=0$, $j=2$}\\
 \mathsf{l}_k, 
 & \parbox[t]{.3\columnwidth}{$i=0$, $j\in\inter{3,N}$}\\
 \mathsf{m}_k, &  \parbox[t]{.3\columnwidth}{$i=0$, $j=N+1$,}\\
\displaystyle -\frac{\mathsf p_k}{3}, & \textnormal{$i,j\in\inter{1,N}$ and $j-i=2$,} \\
\displaystyle \frac{\mathsf q_k}{6}, & \textnormal{$i,j\in\inter{1,N}$ and $j-i> 2$,} \\
\displaystyle -\frac{\gamma}{6} + \frac{11}{18} - \frac{\log(h)}{3}  + \frac{64\log(2) - 54\log(3)}{12} + \frac{\log(2)}{3}  + \frac{1}{6} \psi(\frac 12), & \textnormal{$i\in\inter{1,N-1}$ and $j=i+1$,} \\
\displaystyle -\frac{2}{9}\left[6\log(h) + 3\gamma - 11 + \log(64) - 3\psi(\frac12)\right], & \textnormal{$i\in\inter{1,N}$ and $j=i$.}
\end{cases}
\end{equation}
%
where 
%
\begin{align} \notag
\mathsf{l}_k:= & \frac{1}{3\B{\sqrt{2}}}  \left[-3 j^3 \log (j)+6 j^2 \log (j)+(j-2)^3 (-\log (j-2))+3
   (j-1)^2 (j-2) \log (j-1) \right. \\
   &\left. +(j+1)^2 (j-2) \log (j+1)-2\right] \\
   \mathsf{m}_k:= & \frac{\B{1}}{3}\left[2 (N-3) N^2 \log (N)-N-(N-1)^3 \log (N-1)-(N+1) ((N-4) N+1) \log (N+1)-3\right] \\
\mathsf{p}_k:=&  -24\log(3) - 12\log(4) - 16\log(9) + 27\log(16) + \log(144) \\ \notag
\mathsf{q}_k:=&-(-2 + k)^3\log(-2 + k) + 4(-1 + k)^3\log(-1 + k) - 6 k^3 \log(k)  \\
&+4(1 + k)^3\log(1 + k) - (2 + k)^3\log(2 + k)
\end{align}
and $\psi$ is the digamma function and $\gamma=-\psi(1)$


\section{Numerical evidence and illustrations}\label{sec:numerics}

Trying to characterize a logarithmic convergence rate numerically is a difficult and challenging problem from the numerical perspective. In this section we discuss some partial evidence for the optimality of the logarithmic convergence rate stated in Theorem \ref{main:thm:intro} and we show numerical approximations of some explicit solutions.  

\subsection{On the optimality of the convergence rate}

In general, a good way to observe the optimality of the FEM convergence rate is using the torsion function, namely the solution of the boundary-value problem with right-hand side $f\equiv 1$ (see, among others, \cite[Section 5]{AB17} or \cite[Section 5]{BHS19}). Unfortunately, no closed expression for the torsion function of the logarithmic Laplacian \eqref{t:p:intro} is known.

Because of this, we propose the following. Let $\Omega:=(-1,1)$ and consider the function $u:\R\to\R$ given by
\begin{align}\label{udef}
 u(x)=\frac{1}{\sqrt{-\ln\left(\frac{1-x^2}{2}\right)}} \chi_{\Omega}(x),
\end{align}
where $\chi_{\Omega}$ denotes the characteristic function in the interval $(-1,1)$.

This choice is justified by the following facts
\begin{itemize}
 \item $u$ has a uniformly bounded logarithmic Laplacian in $(-1,1)$ (as a direct nontrivial calculation shows),
 \item has the optimal boundary regularity\footnote{This is needed to observe the optimal convergence rate. Indeed, solutions which are smoother typically yield better convergence rates, see, for instance, \cite[Section 5 and Table 2]{AB17} for the case of the fractional Laplacian.} for logarithmic Dirichlet problems, namely, it behaves as $\ell^{1/2}(d(x))$ (see the Introduction for the definition of $d$ and $\ell$).
\end{itemize}
 Here we only observe these facts numerically (see Figures \ref{fig:1} and \ref{fig:2}), but we will present a detailed proof in a forthcoming paper. See also \cite[Theorem 1.11]{CW19} and \cite[Remark~1.5]{FJW22}.


%%%%%%%%%%%%%%%
%%%%.        Figure       %%%
%%%%%%%%%%%%%%%
% DO NOT ERASE
\begin{figure}[htbp]
	\centering
	%
	\begin{tikzpicture}[scale=1]
	\begin{axis}[xlabel={$x$},xmin=-1.001,xmax=1.001,legend pos=outer north east,
	legend plot pos=left, xtick={-1,0,1},
	legend style={cells={anchor=west},draw=none}]
	%	    
	\pgfplotstableread{./Programas/num_results/file200_lambda1.org}\solution      
	      
	\addplot[thick,color=blue] table[x=xi,y=exsol] \solution; %\label{cost}
	%\addlegendentry{Exact solution};
	
	\end{axis}
	%	
	\end{tikzpicture}

	\caption{Logarithmic laplacian.}
	\label{fig:1}
\end{figure}

Now, consider $h=\frac{1}{N+1}$ for $N\in\mathbb N$, and let $x_i:=-1+ih$ for $i\in\inter{0,N+1}$ so that $-1=x_0<x_1<\ldots<x_{N}<x_{N+1}=1$. Let $f_i:=h L_\Delta u(x_i)$ for $i\in\inter{0,N+1}$, $F:=(f_i)$, and let $\alpha:=A_h^{-1}F$, 
where $A_h$ is the stiffness matrix for the logarithmic Laplacian given in \eqref{stiff:LogLap}. Let 
\begin{align}\label{vdef}
v_h(x):=\sum_{i=0}^{N+1}\alpha_i \phi_i(x), 
\end{align}
which is a good approximation for $u_h$ (defined in Theorem \ref{main:thm:intro}), see Figure \ref{fig:2}.

%%%%%%%%%%%%%%%
%%%%.        Figure       %%%
%%%%%%%%%%%%%%%
% DO NOT ERASE
\begin{figure}[h!]
	\centering
	%
	\begin{tikzpicture}[scale=1]
	\begin{axis}[xlabel={$x$},xmin=-1.001,xmax=1.001,legend pos=outer north east,
	legend plot pos=left, xtick={-1,0,1},
	legend style={cells={anchor=west},draw=none}]
	%	    
	\pgfplotstableread{./Programas/num_results/sol-log_num-vs-exact_20-06-2023_12h48.org}\solution   
	\pgfplotstableread{./Programas/num_results/sol-log_num-vs-exact_21-06-2023_16h18.org}\solutionfine         
	      
	\addplot[thick,color=blue] table[x=xi,y=exsol] \solutionfine; %\label{cost}
	 \addlegendentry{$u$};
	\addplot[thick,color=red,every mark/.append style={solid, fill=none}, mark=x] table[x=xi,y=numsol] \solution; %\label{cost}
	 \addlegendentry{$v_h$};
	
	\end{axis}
	%	
	\end{tikzpicture}

	\caption{A comparison between $u$ given by \eqref{udef} and $v$ given by \eqref{vdef} for $N=50$.}
	\label{fig:2}
\end{figure}

As seen in Figure \ref{fig:2}, the largest error ocurrs close to the boundary of the interval, where the function $u$ has a logarithmic behavior.  To quantify this error we compute the following quantities,
\begin{align*}
 a_h&:=\left(\int_{-1}^1 |u-v_h|^2\, dx\right)^\frac{1}{2}\qquad \text{(the $L^2$-norm of the error between $u$ and $v$)},\\
 b_h&:=\left(\int_{-1+h}^{1-h} |u-v_h|^2\, dx\right)^\frac{1}{2}\qquad \text{(a local $L^2$-norm of the error between $u$ and $v$)},\\
 c_h&:=\sup_{(-1,1)} |u-v_h|\qquad \text{(the $L^\infty$-norm of the error between $u$ and $v$)}.
\end{align*}

We remark that $b_h$ gives information on the error rate in the interior of the interval, slightly far away from the endpoints. 

We are particularly interested in exhibiting evidence for the optimality of the logarithmic rate in Theorem \ref{main:thm:intro}.  For this, we compute these quantities for different values of $h$ and describe the behavior of the error curve, which we show in Figure \ref{fig:3}.

%%%%%%%%%%%%%%%
%%%%.        Figure       %%%
%%%%%%%%%%%%%%%
% DO NOT ERASE
\begin{figure}[h!]
	\centering
	%
	\subfloat[The $L^2$-norm $(a_h)$.]{
	\begin{tikzpicture}[scale=0.8]
	\begin{loglogaxis}[xlabel={$h$},legend pos=outer north east,
	legend plot pos=left,
	legend style={cells={anchor=west},draw=none}]
	%	    
	\pgfplotstableread{./Programas/num_results/sol-log_convergence_data_20-06-2023_13h27.org}\solution      
	      
	\addplot[color=black,every mark/.append style={solid}, mark=square*] table[x=h,y=L2norm] \solution; %\label{cost}
	% \addlegendentry{Convergence rate in the norm $\mathbb H$};
	
	\end{loglogaxis}\label{l2_ah}
	%	
	\end{tikzpicture}
	} \quad 
	\subfloat[The $L_{loc}^2$-norm $(b_h)$.]{
	\begin{tikzpicture}[scale=0.8]
	\begin{loglogaxis}[xlabel={$h$},legend pos=outer north east,
	legend plot pos=left,
	legend style={cells={anchor=west},draw=none}]
	%	    
	\pgfplotstableread{./Programas/num_results/sol-log_convergence_data_20-06-2023_13h27.org}\data      
	      
	\addplot[color=black,every mark/.append style={solid}, mark=square*] table[x=h,y=L2locnorm] \data; %\label{cost}
	% \addlegendentry{Convergence rate in the norm $\mathbb H$};
	
	\end{loglogaxis}
	%	
	\end{tikzpicture}
	}
	\\\bigskip
	\subfloat[The $L^\infty$-norm $(c_h)$.]{
	\begin{tikzpicture}[scale=0.8]
	\begin{loglogaxis}[xlabel={$h$},legend pos=outer north east,
	legend plot pos=left,
	legend style={cells={anchor=west},draw=none}]
	%	    
	\pgfplotstableread{./Programas/num_results/sol-log_convergence_data_21-06-2023_11h34.org}\data      
	      
	\addplot[color=black,every mark/.append style={solid}, mark=square*] table[x=h,y=Linfnorm] \data; %\label{cost}
	% \addlegendentry{Convergence rate in the norm $\mathbb H$};
	
	\end{loglogaxis}
	%	
	\end{tikzpicture}
	}
	\caption{Error curves for $a_h$, $b_h$, and $c_h$.}
	\label{fig:3}
\end{figure}

Note that these curves are not straight lines. This contrasts with the typical behavior of algorithms with convergence rate of order $h^p$ for some $p>0$; in particular, this is the case of the fractional Laplacian, see for instance \cite[Figure 8]{BH17}, \cite[Figure 1]{AB17}, and Table \ref{table:frac_lap} in the Appendix.  This suggests that the optimal convergence rate for the logarithmic Laplacian is \emph{not} of power-type, as suggested in Theorem \ref{main:thm:intro}.

Although the curve in Figure \ref{l2_ah} looks almost straight, a computation of its slope (with respect to consecutive points) shows otherwise, as presented in Table \ref{table:1}.


\begin{table}[h!]
\centering
%
\pgfplotstableread{./Programas/num_results/sol-log_convergence_data_21-06-2023_11h34.org}\data
%
\pgfplotstableset{
columns={N,h,L2norm,slopeL2,L2locnorm,slopeL2loc,Linfnorm,slopeinf},
%%%
columns/N/.style={
column name=$N$,
},	
%%% 
columns/h/.style={
column name=$h$,
sci,sci zerofill,sci subscript,
precision=2,
column type/.add={}{|}},	
%%%
columns/L2norm/.style={
column name=$L^2$--norm ($a_h$),
sci,sci zerofill,sci subscript,
precision=3},
%%%
columns/slopeL2/.style={
column name=slope,
sci,sci zerofill,sci subscript,
precision=2,
column type/.add={}{|}},	
%%%
columns/L2locnorm/.style={
column name=$L^2_{loc}$-norm ($b_h$),
sci,sci zerofill,sci subscript,
precision=3},
%%%
columns/slopeL2loc/.style={
column name=slope,
%sci,sci zerofill,sci subscript,
precision=2,
column type/.add={}{|}},	
%%%
columns/Linfnorm/.style={
column name=$L_{\infty}$--norm ($c_h$),
sci,sci zerofill,sci subscript,
precision=4,
column type/.add={}{}},
%%%
columns/slopeinf/.style={
column name=slope,
sci,sci zerofill,sci subscript,
precision=2,
column type/.add={}{}},
}
%
\pgfplotstabletypeset[clear infinite, empty cells with={\ensuremath{-}},
every head row/.style={
before row=\toprule,after row=\midrule},
every last row/.style={
after row=\bottomrule},
]{\data}
%
\caption{Error data  for $a_h$, $b_h$, and $c_h$.}
\label{table:1}
\end{table}

We mention that this behavior is robust with respect to the size of the interval.  Below we present the same experiment for the function 
$w(x)=\left(
-\ln\left(\frac{1}{2}-2x^2\right)\right)^{-\frac{1}{2}} \chi_{(-\frac{1}{2},\frac{1}{2})}(x)$, obtaining again nonconstant decreasing slopes, see Table \ref{table:2}. 



\begin{table}[h!]
\centering
%
\pgfplotstableread{./Programas/num_results/sol-log_convergence_data_21-06-2023_16h57.org}\data
%
\pgfplotstableset{
columns={N,h,L2norm,slopeL2,L2locnorm,slopeL2loc,Linfnorm,slopeinf},
%%%
columns/N/.style={
column name=$N$,
},	
%%% 
columns/h/.style={
column name=$h$,
sci,sci zerofill,sci subscript,
precision=2,
column type/.add={}{|}},	
%%%
columns/L2norm/.style={
column name=$L^2$--norm,
sci,sci zerofill,sci subscript,
precision=3},
%%%
columns/slopeL2/.style={
column name=slope,
sci,sci zerofill,sci subscript,
precision=2,
column type/.add={}{|}},	
%%%
columns/L2locnorm/.style={
column name=$L^2_{loc}$-norm,
sci,sci zerofill,sci subscript,
precision=3},
%%%
columns/slopeL2loc/.style={
column name=slope,
%sci,sci zerofill,sci subscript,
precision=2,
column type/.add={}{|}},	
%%%
columns/Linfnorm/.style={
column name=$L_{\infty}$--norm,
sci,sci zerofill,sci subscript,
precision=4,
column type/.add={}{}},
%%%
columns/slopeinf/.style={
column name=slope,
sci,sci zerofill,sci subscript,
precision=2,
column type/.add={}{}},
}
%
\pgfplotstabletypeset[clear infinite, empty cells with={\ensuremath{-}},
every head row/.style={
before row=\toprule,after row=\midrule},
every last row/.style={
after row=\bottomrule},
]{\data}
%
\caption{Error data for the approximation of $w$.}
\label{table:2}
\end{table}


\begin{remark}
Let $u$ as in \eqref{udef} and $v$ as in \eqref{vdef}.  Let $u_h$ be the FEM approximation of $u$ given by Theorem \ref{main:thm:intro}.  Then,
\SJ{Constant from the Poincar\'e inequality is missing}
\begin{align*}
\|u-u_h\|^2_{\mathbb H(\Omega)}
\geq
\|u-u_h\|^2_{L^2(\Omega)}
=\|u-v_h\|^2_{L^2(\Omega)}
+2\int_\Omega  (u-v_h)(v_h-u_h)\, dx
+\|v_h-u_h\|^2_{L^2(\Omega)}.
\end{align*}
In Table \ref{table:1}, we see evidence that the norm $a_h^2:=\|u-v_h\|^2_{L^2(\Omega)}$ does not decay as $h^p$ for any $p>0$. We do not have an exact estimate for the norm $R_h:=\|v_h-u_h\|^2_{L^2(\Omega)}$, but this is not important to establish that the norm $E_h:=\|u-u_h\|^2_{\mathbb H(\Omega)}$ does not decay as a power.  Indeed, if the term $R_h=O(h^p)$ for some $p>0$, then $E_h=O(a_h^2)$, which does not decay as a power.  On the other hand, if $R_h$ has a worse decay than $a_h^2$, then this decay dominates and therefore $E_h$ does not decay as power also in this case.  Summarizing,   Table \ref{table:1} does present relevant evidence on the optimality of the type of rate shown in Theorem \ref{main:thm:intro}.
\end{remark}


\subsection{Approximation of explicit solutions}

Now we follow the approach in \cite{L22} and use three explicit solutions to further illustrate the FEM approximation of solutions to logarithmic Dirichlet problems. For $r>0$, let  $\Omega:=(-L,L)$, $h_\Omega(x):=-\ln(L^2-|x|^2)$, and $\rho_1:=2 \ln (2)+\psi\left(\frac{1}{2}\right)-\gamma\approx -1.15443$, where $\gamma=-\Gamma^{\prime}(1)$ is the Euler-Mascheroni constant and $\psi$ is the Digamma function given by $\psi=\frac{\Gamma^{\prime}}{\Gamma}$.  For $x\in\R$, let 
\begin{align*}
u_1(x):=\chi_{[-L,L]}(x), \qquad
u_2(x):=\chi_{[-L,L]}(x) x, \qquad
u_3(x):=(L^2-x^2)_+=(L^2-x^2)\chi_{[-L,L]}(x).
\end{align*}
Then, for $x\in \Omega$,
\begin{align*}
    L_\Delta u_1(x)&=h_\Omega(x)+\rho_1,\\
    L_\Delta u_2(x)&=x(2+h_\Omega(x)+\rho_1),\\
    L_\Delta u_3(x)&=L^2-3x^2+(h_\Omega(x)+\rho_1)(L^2-x^2).
\end{align*}

By \cite[Corollary 1.9]{CW19} it is stated that the (weak) maximum principle holds in 1D whenever 
\begin{align*}
    2L=|\Omega|\leq 2 e^{\frac{1}{2}(\psi(\frac{1}{2})-\gamma)}|B_1(0)|=4e^{\frac{1}{2}(\psi(\frac{1}{2})-\gamma)},
    \end{align*}
    that is,
    \begin{align*}
        L\leq 2 e^{\frac{1}{2}(\psi(\frac{1}{2})-\gamma)}\approx 0.6453.
\end{align*}

%%%%%%%%%%%%%%%
%%%%.        Figure       %%%
%%%%%%%%%%%%%%%
% DO NOT ERASE
\begin{figure}[h!]
	\centering
	%
	\subfloat[The function $u_1$ and its FEM approximation.]{
	\begin{tikzpicture}[scale=0.8]
	\begin{axis}[xlabel={$x$},xmin=-1,xmax=1,legend pos=outer north east,
	legend plot pos=left, xtick={-1,0,1},
	legend style={cells={anchor=west},draw=none}]
	%	    
	\pgfplotstableread{./Programas/num_results/sol-log_num-vs-exact_20-06-2023_16h03.org}\solution      
	      
	\addplot[thick,color=blue] table[x=xi,y=exsol] \solution; %\label{cost}
	\addplot[thick,color=red,every mark/.append style={solid, fill=none}, only marks, mark=x] table[x=xi,y=numsol] \solution; %\label{cost}
	
	\end{axis}
	%	
	\end{tikzpicture}
	}
	%
	\subfloat[The function $u_2$ and its FEM approximation.]{
	\begin{tikzpicture}[scale=0.8]
	\begin{axis}[xlabel={$x$},xmin=-1,xmax=1,ymin=0.0,legend pos=outer north east,
	legend plot pos=left, xtick={-1,0,1},
	legend style={cells={anchor=west},draw=none}]
	%	    
	\pgfplotstableread{./Programas/num_results/sol-log_num-vs-exact_20-06-2023_16h01.org}\solution      
	      
	\addplot[thick,color=blue] table[x=xi,y=exsol] \solution; %\label{cost}
	\addplot[thick,color=red,every mark/.append style={solid, fill=none}, mark=x] table[x=xi,y=numsol] \solution; %\label{cost}
	
	\end{axis}
	%	
	\end{tikzpicture}
	} \\
	\subfloat[The function $u_3$ and its FEM approximation.]{
	\begin{tikzpicture}[scale=0.8]
	\begin{axis}[xlabel={$x$},xmin=-1,xmax=1,legend pos=outer north east,
	legend plot pos=left, xtick={-1,0,1},
	legend style={cells={anchor=west},draw=none}]
	%	    
	\pgfplotstableread{./Programas/num_results/sol-log_num-vs-exact_20-06-2023_15h59.org}\solution      
	      
	\addplot[thick,color=blue] table[x=xi,y=exsol] \solution; %\label{cost}
	\addlegendentry{Exact solution};
	\addplot[thick,color=red,every mark/.append style={solid, fill=none}, only marks, mark=x] table[x=xi,y=numsol] \solution; %\label{cost}
	 \addlegendentry{Numerical approximation};
	
	\end{axis}
	%	
	\end{tikzpicture}
	} 
	\caption{Exact solutions and their numerical approximations for $N=50$.}
	\label{fig:comp_sol}
\end{figure}

We remark that the functions $u_1$ and $u_2$ do not satify the assumptions of Theorem~\ref{main:thm:intro}, because the right-hand side does not belong to $\mathcal Y(\Omega)$. Nevertheless, the FEM yields a good approximation also in these cases.  On the other hand, the function $u_3$ is Lipschitz in $\R$ and thus it exhibits a better convergence rate. Therefore, we cannot use $u_3$ to show the optimality of the convergence rate stated in Theorem \ref{main:thm:intro}.


\section{Approximation of the eigenvalues of the logarithmic Laplacian}

Standard Fredholm theory implies that if $0$ is not an eigenvalue of $L_\Delta$, then the Dirichlet problem with $L_{\Delta}$ is uniquely solvable for any $f\in L^2(\Omega)$. By the regularity theory, this translates to classical solutions in view of  Theorem \ref{eq:regularity}. To approximate the eigenvalues of $L_{\Delta}$ in $\Omega$, we consider first the case of Theorem \ref{eq:regularity}(ii), that is, the case in which $0$ is an eigenvalue of $L_{\Delta}$ in $\Omega$. In this case there is a nontrivial $\phi\in \mathcal X^{\alpha}(\Omega)$ satisfying $L_{\Delta}\phi=0$ in  $\Omega$ and $\phi=0$ in $\R\setminus \Omega$ ---we emphasize here that the following argument can be done in any dimension. Note that if we define for $r>0$ a function $\phi_r=\phi(\cdot/r)$, then $L_{\Delta}\phi_r=-2\ln(r)\phi_r$ in $r\Omega$ and $\phi_r=0$ in $\R\setminus r\Omega$. Thus having $0$ as an eigenvalue for $\Omega$ gives us the eigenvalue $-2\ln(r)$ in $r\Omega$.\\

In this way, if we can approximate such $L>0$ such that the operator $L_{\Delta}$ has $0$ as an eigenvalue in $\Omega=(0,L)$, we can classify from there the eigenvalues of $(0,\tilde{L})$ for a general $\tilde{L}>0$. To find these values of $L$, we first show that the eigenvalues of the stiffness matrix $\mathcal A_h^L$ actually converge to the eigenvalues of $L_{\Delta}$ and then we use the condition of the stiffness $\mathcal{A}^L_h$ to find the cases in which there is zero as an eigenvalue. Here, we denote the condition of a matrix $A$ by
\begin{equation}\label{defi:condition}
\textnormal{cond}(A):=\Big|\frac{\lambda_{\max}(A)}{\lambda_{\min}(A)}\Big|,
\end{equation}
where with $\sigma(A)=\{\lambda\in \R\;:\; \lambda\text{ is an eigenvalue of $A$}\}$ for a matrix $A$ we denote
$$
\lambda_{\max}(A):=\max_{\lambda\in \sigma(A)}|\lambda|\quad\text{ and }\quad \lambda_{\min}(A):=\min_{\lambda\in \sigma(A)}|\lambda|.
$$
In this way, $\textnormal{cond}(\mathcal{A}^L_h)$ blows up while moving $L$ from $0$ to infinity every time there is $0$ as an eigenvalue of $\mathcal{A}^L_h$.\\

Since we already know that there is an infinite sequence of eigenvalues $\lambda_1<\lambda_2\leq\ldots\lambda_k\to \infty$ for $k\to\infty$ of $L_{\Delta}$ in $\Omega=(0,L)$ for some fixed $L$ by \cite[Theorem 3.4]{CW19}, it follows that we can find a sequence of values $L_i>0$, $i\in \N$ satisfying $0<L_1<L_2<\ldots$ with $L_i\to\infty$ for $i\to \infty$ such that $0$ is an eigenvalue of $L_{\Delta}$ in $(0,L_i)$. With the above it then follows that the eigenvalues of $L_{\Delta}$ in $(0,L)$ are given by $\lambda_i=2\ln(L_i/L)$ for $i\in \N$ (without multiplicity).\\

We first show the convergence of the eigenvalues of $\mathcal{A}^L_h$ to eigenvalues of $L_{\Delta}$ for $h\to 0$. Then we present an approximation of $\textnormal{cond}(A^L_h)$ for small enough $h$ and give the approximations of the eigenvalues in $(0,L)$ for general $L>0$.\\

For the convergence, we cannot work with Theorem \ref{main:thm:intro}, since it is not clear whether the eigenfunctions belong to the space $\mathcal{Y} (\Omega)$. We first classify the eigenvalues $\{\lambda_k\}_{k\in \N}$ of $L_{\Delta}$ with the Courant-Fischer minimax principle, see e.g. \cite[Proposition 2.3 and Remark 2.4]{GJW22}, that is it holds
$$
\lambda_k=\inf_{\substack{M\subset \mH(\Omega)\\ \dim(M)=k}} \max_{\substack{u\in M\setminus\{0\}\\ \|u\|_{L^2(\Omega)}=1}} \mathcal{E}_L(u,u),\quad k\in \N,
$$
where $U\subset\mH(\Omega)$ is any dense subspace of $\mH(\Omega)$.

\begin{proposition}\label{prop:eigen-approx}
It holds
$$
\lambda_{k,h}:=\inf_{\substack{M\subset \mathcal{V}_{h}\\ \dim(M)=k}} \max_{\substack{u\in M\setminus\{0\}\\ \|u\|_{L^2(\Omega)}=1}} \mathcal{E}_L(u,u)\to \lambda_k\quad\text{for $h\to0$.}
$$
Here, $\lambda_{k,h}$ equals to the $k$-the eigenvalue of $\mathcal{A}^L_{h}$.
\end{proposition}
\begin{proof}
First note that by the Courant-Fischer minimax principle it holds, for $h>0$,
$$
\lambda_k=\inf_{\substack{M\subset \mH(\Omega)\\ \dim(M)=k}} \max_{\substack{u\in M\setminus\{0\}\\ \|u\|_{L^2(\Omega)}=1}} \mathcal{E}_L(u,u)\leq \inf_{\substack{M\subset \mathcal{V}_{h}\\ \dim(M)=k}} \max_{\substack{u\in M\setminus\{0\}\\ \|u\|_{L^2(\Omega)}=1}} \mathcal{E}_L(u,u)=\lambda_{k,h}.
$$
For the other inequality, first recall that by the Poincar\'e inequality, we fix a constant $C>0$ such that 
$$
\|u\|_{L^2(\Omega)}\leq C\|u\|_{\mH(\Omega)}\quad\text{for all $u\in \mH(\Omega)$.}
$$
Let  $\delta>0$. Then there is a subspace $M_{\delta}\subset \mH(\Omega)$ with $\dim(M_{\delta})=k$ and some $m\in M_{\delta}$ with $\|m\|_{L^2(\Omega)}=1$ such that
$$
\max_{\substack{ u\in M_{\delta}\setminus \{0\}\\ \|u\|_{L^2(\Omega)}=1}} \mathcal{E}_L(u,u)=\mathcal{E}_{L}(m,m)\leq \lambda_k+\frac{\delta}{2}.
$$
Using Theorem \ref{density:thm} we find $h_0>0$ such that 
$$
\|m-m_h\|_{\mH(\Omega)}\leq \frac{\delta}{2C} \quad\text{for all $h\in(0,h_0)$}
$$
and it holds
$$
0<\|m_h\|_{L^2(\Omega)}\leq \|m-m_H\|_{L^2(\Omega)}+1\leq \frac{\delta}{2}+1
$$
Then
\begin{align}
\lambda_{k}&\leq \lambda_{k,h}\leq \|m_h\|_{L^2(\Omega)}^2 \mathcal{E}_L(m_h,m_h)\notag\\
&\leq ( \frac{\delta}{2}+1)^2\Big(\mathcal{E}_L(m_h-m,m_h+m)+\mathcal{E}_L(m,m)\Big)\notag\\
&\leq ( \frac{\delta}{2}+1)^2\big(\lambda_k+\delta\big)+ ( \frac{\delta}{2}+1)^2\mathcal{E}_L(m_h-m,m_h+m).\label{lambda-k-upperbound}
\end{align}
Here, by the definition of $\mathcal{E}_L$ and the Cauchy-Schwarz inequality
\begin{align}
\mathcal{E}_L(m_h-m,m_h+m)&\leq \mathcal{E}(m_h-m,m_h+m)+\|m_h-m\|_{L^1(\R^N)}\|m\|_{L^1(\R^N)}+\rho_1\|m_h-m\|_{L^2(\R^N)}\notag\\
&\leq \|m_h-m\|_{\mH(\Omega)}\Big(\|m_h\|_{\mH(\Omega)}+\|m\|_{\mH(\Omega)}\Big)+\Big(|\Omega|+\rho_1\Big)\|m_h-m\|_{L^2(\R^N)}\notag\\
&\leq \frac{\delta}{2C}\Big(\|m_h\|_{\mH(\Omega)}+\|m\|_{\mH(\Omega)}+C\big(|\Omega|+\rho_1\big)\Big).\label{lambda-k-upperbound2}
\end{align}
Since $\|m_h\|_{\mH(\Omega)}\to \|m\|_{\mH(\Omega)}$ for $h\to 0$, we can assume that we have
$$
\|m_h\|_{\mH(\Omega)}\leq \|m\|_{\mH(\Omega)}+1
$$
by assuming $h_0$ smaller if necessary. Thus from \eqref{lambda-k-upperbound} and \eqref{lambda-k-upperbound2} we have
\begin{align*}
\lambda_k\leq \lambda_{k,h}\leq ( \frac{\delta}{2}+1)^2\big(\lambda_k+\delta\big)+ \delta\frac{(\delta+2)^2}{8C}\Big(2\|m\|_{\mH(\Omega)}+1+C\big(|\Omega|+\rho_1\big)\Big).
\end{align*}
Sending $\delta\to 0$, the right-hand side converges to $\lambda_k$ and this shows the first part. For the last part, note that since the $\phi_i$, $i\in\inter{0,N+1}$ form a basis of $\mathcal{V}_h$, the dimension of $\mathcal{V}_{h}$ is $N+2$ and, for any $u\in V_{\eps}$, we can find $r_k\in \R$, $k\in \inter{0,N+1}$ such that
$$
u=\sum_{k=0}^{N+1}r_k\phi_k\quad\text{and thus}\quad \mathcal{E}_L(u,u)=\sum_{k,i=0}^{N+1}r_kr_i\mathcal{E}_L(\phi_i,\phi_j)= \mathcal{A}^L_\eps r\cdot r.
$$
From here, it is easy to see that $\lambda_{k,h}$ denotes the $k$-the eigenvalue of $\mathcal{A}^L_{h}$.
\end{proof}



\SJc{ToDos:}
\begin{enumerate}
\item \SJc{Add picture ``condition vs. values of $L$'' ---for comparison: picture 17.1 in \cite{L22} }
\item \SJc{Add a table where by the Golden Section Search (see below) different $L_i$ in certain intervals and depending on $h$ are approximated ---for comparison: table 17.1 in \cite{L22} }
\item \SJc{Add a corollary with a table of the different approximations of the eigenvalues of $L_{\Delta}$ in $(0,L)$ ---for comparison: Proposition 17.1 in \cite{L22} }
\end{enumerate}
\SJc{In \cite{L22} the Golden Section Search was used according to the book \emph{Numerical Recipes} by Press, William H., Teukolsky, Saul A., Vetterling, William T. and Flannery, Brian P. (2007)}\\
\SJc{the numerics in \cite{L22} are done with $r$ such that $\Omega=(-r,r)$ so that the values here will be different}

\begin{remark}
We emphasize that the scaling of the logarithmic Laplacian behaves the same in all dimensions. Due to this, and by following the proof of Proposition \ref{prop:eigen-approx} closely, the approach to estimate the eigenvalues of $L_{\Delta}$ in an arbitrary domain $\Omega\subset\R^N$ can be done in an equivalent way, once the stiffness matrix is calculated and Theorem \ref{density:thm} is verified for the corresponding approximation space $\mathcal{V}_h$.
\end{remark}


 \section{Open problems}\label{sec:o:p}

 To close this paper, we discuss some open problems.
 \begin{enumerate}
    \item \emph{Adaptive mesh:} In this paper we only considered a homogeneous mesh.  It would be interesting to study the effect that an adaptive mesh would have in the convergence rate.
     \item \emph{Dimension 2 and 3:} We think that once the one-dimensional case has been established, it would be interesting to study the convergence of FEM for logarithmic problems in dimensions 2 and 3.
 \end{enumerate}



\appendix

\section{Some auxiliary lemmas}
%
\begin{lemma}\label{lem:over}
Let $R>0$ and $\alpha\in(0,1)$. Then there is $C=C(R,\alpha)>0$ such that 
%
\begin{equation}
    \int_0^{R} \frac{\ell^{1+\alpha}(\rho)}{\rho}d\rho\leq C.
\end{equation}
%
\end{lemma}
%
\begin{proof}
Let $r\in(0,0.1)$ be fixed and assume that $R>0.1$. We split the integral into two parts, that is
%
\begin{equation*}
    \int_{0}^{R}\frac{\ell^{1+\alpha}(\rho)}{\rho}d\rho=\int_{0}^{r} \frac{\ell^{1+\alpha}(\rho)}{\rho}d\rho + \int_{r}^{R}\frac{\ell^{1+\alpha}(\rho)}{\rho}d\rho
\end{equation*}
%
Using that $d\rho/\rho=d(\ln \rho)$ and by definition of the function $\ell$, we have
%
\begin{align*}
    \int_{0}^{R}\frac{\ell^{1+\alpha}(\rho)}{\rho}d\rho &=-\int_{0}^{r} \frac{d|\ln(\rho)|}{|\ln(\rho)|^{1+\alpha}} + \int_{r}^{R}\frac{\ell^{1+\alpha}(\rho)}{\rho}d\rho 
    \\ &\leq -\int_{0}^{r} \frac{d|\ln(\rho)|}{|\ln(\rho)|^{1+\alpha}} + \int_{r}^{R}\frac{|\ln(r)|^{-1-\alpha}}{\rho}d\rho \\
    &=\frac{\ell^\alpha(r)}{\alpha}+\ln(r)^{-1-\alpha}\ln(R/r):=C(R,\alpha).
\end{align*}
%
The case $R\leq 0.1$ follows similarly. This ends the proof. 
\end{proof}

\begin{lemma}\label{lem:ellbeta}
For every $R>0$ and $\beta>0$ there is $C=C(R,\beta)>0$ such that
\begin{align*}
\int_0^R \ell^{-\beta}(\rho)\, d\rho<C.
\end{align*}
\end{lemma}
\begin{proof}
 It is well known that the Gamma function has the following integral representation
 \begin{align*}
     \Gamma(z)=\int_0^1\left(\ln \frac{1}{t}\right)^{z-1}\, dt\qquad \text{ for }z>0.
 \end{align*}
Then, using \eqref{ell:def},
\begin{align*}
\int_0^R \ell^{-\beta}(\rho)\,d\rho
=
\int_0^{\rho_0} |\ln(x)|^{\beta}\,dx
+
\frac{R}{\ell^{\beta}(\rho_0)}
&\leq 
\int_0^{1} |\ln(x)|^{\beta}\,dx
+
\frac{R}{\ell^{\beta}(\rho_0)}=\Gamma(1+\beta)+\ell^{-\beta}(\rho_0)R.
\end{align*}
\end{proof}


\begin{lemma}\label{lem:int:bds}
Let $\alpha>0$ and $\Omega=(0,L)$ for some $L>0$. There is $C=C(L,\alpha)>0$ such that 
\begin{align*}
\int_{\Omega}\int_{\Omega}\frac{\ell^{1+\alpha}(|x-y|)}{|x-y|\ell^{2+2\alpha}(d(x,y))}\,dy\, dx+ \int_{\Omega}\ell^{-1+\alpha}(d(x))\, dy\,dx<C.
\end{align*}
\end{lemma}
\begin{proof}
Recall that $d(x):=\dist(x,\partial \Omega)$ and $d(x,y)=\min(d(x),d(y)).$ Then
\begin{align*}
\int_0^L \ell^{-1+\alpha}(d(x))\,dx&\leq \int_0^L \ell^{-1+\alpha}(x)\,dx+\int_0^L \ell^{-1+\alpha}(L-x)\,dx=:C_1<\infty,
\end{align*}
by Lemma \ref{lem:ellbeta}. On the other hand,
\begin{align*}
\int_0^L\int_0^L\frac{\ell^{1+\alpha}(|x-y|)}{|x-y|\ell^{2+2\alpha}(d(x,y))}\,dy\, dx
&\leq 2\int_0^L\int_0^L\frac{\ell^{1+\alpha}(|x-y|)}{|x-y|\ell^{2+2\alpha}(d(x))}\,dy\, dx\\
&\leq 4\int_0^L\left(\int_0^L\frac{\ell^{1+\alpha}(|x-y|)}{|x-y|}\,dy\right)\frac{1}{\ell^{2+2\alpha}(x)}\, dx\\
&\leq 4\left(\int_0^L\frac{\ell^{1+\alpha}(\rho)}{\rho}\,d\rho\right)\int_0^L\frac{1}{\ell^{2+2\alpha}(x)}\, dx=:C_2<\infty,
\end{align*}
by Lemmas \ref{lem:over} and \ref{lem:ellbeta}.
\end{proof}







The next lemma follows closely the proof of \cite[Theorem 1.1]{CW19}.
\begin{lemma}\label{phi0:conv}
For $1<p<\infty$,
\begin{align*}
\lim_{s\to 0}\left\|\frac{(-\Delta)^s \varphi_0-\varphi_0}{s}-L_\Delta \varphi_0    \right\|_{L^p(\R)}
=\lim_{s\to 0}\left\|\frac{(-\Delta)^s \varphi_{N+1}-\varphi_{N+1}}{s}-L_\Delta \varphi_{N+1}    \right\|_{L^p(\R)}
=0.
\end{align*}
\end{lemma}
\begin{proof}
It suffices to show the claim for $\varphi_0$. Recall that $\Omega=(0,L)$ and $\varphi_0(x)=2(1-\frac{x}{h})\chi_{[0,h]}$ for $h\in(0,1)$ small (with respect to $L$).  Let $R>4.$  For $x \in \mathbb{R}$, we have
$$
\left[(-\Delta)^s \varphi_0\right](x)=\int_{\mathbb{R}} \frac{\varphi_0(x)-\varphi_0(x+z)}{|z|^{1+2s}} d z=A_R(s, x)+D_R(s) \varphi_0(x)
$$
with
$$
A_R(s, x):=c_{1, s}\left(\int_{B_R} \frac{\varphi_0(x)-\varphi_0(x+z)}{|z|^{1+2 s}} d z-\int_{\mathbb{R} \backslash B_R} \frac{\varphi_0(x+z)}{|z|^{1+2s}} d z\right)
$$
and $D_R(s):=c_{1, s} \int_{\mathbb{R} \backslash B_R}|z|^{-1-2 s} d z=\frac{c_{1, s}}{s}R^{-2 s}.$ Let $c_{1, s}=s d_1(s)$ with $d_1(s):=\frac{c_{1, s}}{s}=\pi^{-\frac{1}{2}} 2^{2 s} \frac{\Gamma\left(\frac{1}{2}+s\right)}{\Gamma(1-s)}.$  Note that, for $x\in B_R\backslash \{0\}$,
\begin{align*}
\frac{A_R(s, x)}{s} & =d_1(s) \int_{B_R} \frac{\varphi_0(x)-\varphi_0(x+z)}{|z|^{1+2 s}} d z
\rightarrow \tilde{A}_R(x):=\int_{B_R} \frac{\varphi_0(x)-\varphi_0(x+z)}{|z|} d z \quad \text { as } s \rightarrow 0^{+}
\end{align*}
These integrals are singular at $x=0$, but the singularity is of logarithmic type.  As a consequence, direct computations show that this convergence also holds in $L^p(\R)$ for $1<p<\infty$.  The rest of the argument follows exactly as in \cite[Theorem 1.1]{CW19}.
\end{proof}

\begin{lemma}\label{ibyp:lem}
For $i,j=0,\ldots,N+1$ and $s<\frac{1}{4}$, we have that
\begin{align*}
    \cE_s(\varphi_i,\varphi_j)&=\int_{\R}(-\Delta)^s\varphi_i\varphi_j\, dx,\qquad \cE_L(\varphi_i,\varphi_j)=\int_{\R}L_\Delta\varphi_i\varphi_j\, dx.
\end{align*}
\end{lemma}
\begin{proof}The proof follows by a standard application of Fubini's theorem. 
\end{proof}


\section{Convergence rates for fractional problems}
 
The function
\begin{align*}
 U(x)=\frac{(1-x^2)^s}{\Gamma(1+2s)}\chi_{(-1,1)}
\end{align*}
is the torsion function for the fractional Laplacian, namely, $(-\Delta)^s U(x)=1$ for $x\in (-1,1)$.  Using the stiffness matrix given in \eqref{stiff:frac}, we obtain the following table for comparison purposes (cf. Table \ref{table:1}).  We mention that these rates are optimal, as shown in \cite{Bor17} (see also \cite[Section 5]{AB17} and \cite[Section 5]{BHS19}).
 
 
{\color{white}.}
\begin{table}[h!]
\centering
%
\pgfplotstableread{./Programas/num_results/sol-log_convergence_data_20-06-2023_16h35.org}\data
%
\pgfplotstableset{
columns={N,h,L2norm,slopeL2,Hsnorm,slopeHs},
%%%
columns/N/.style={
column name=$N$,
},	
%%% 
columns/h/.style={
column name=$h$,
sci,sci zerofill,sci subscript,
precision=2,
column type/.add={}{|}},	
%%%
columns/L2norm/.style={
column name=$L^2$--norm,
sci,sci zerofill,sci subscript,
precision=3},
%%%
columns/slopeL2/.style={
column name=slope,
sci,sci zerofill,sci subscript,
precision=2,
column type/.add={}{|}},	
%%%
columns/Hsnorm/.style={
column name=$H^s$-norm,
sci,sci zerofill,sci subscript,
precision=3},
%%%
columns/slopeHs/.style={
column name=slope,
%sci,sci zerofill,sci subscript,
precision=2,
column type/.add={}{}},	
%%%
}
%
\pgfplotstabletypeset[clear infinite, empty cells with={\ensuremath{-}},
every head row/.style={
before row=\toprule,after row=\midrule},
every last row/.style={
after row=\bottomrule},
]{\data}
%
\caption{Convergence data fractional Laplacian.}
\label{table:frac_lap}
\end{table}
 
 
 \bibliographystyle{plain}
 \bibliography{biblio}

Citar los artículos:
\begin{itemize}
 \item De Chhetri que cita a victor
 \item artículos de Juan Pablo
\item artículo de Nochetto con otro interpolador
 \item artículos de Feulefack y Remi
\end{itemize}



\end{document}

