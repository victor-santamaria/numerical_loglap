\documentclass[a4paper,10pt]{article}

\usepackage{amsfonts}
\usepackage{amsmath}
\usepackage[utf8]{inputenc}
\usepackage[english]{babel}
\usepackage[english]{varioref}
\usepackage{microtype}
\usepackage{fullpage}

\DisableLigatures{encoding = *, family = * }

\pagestyle{empty}

\newcommand{\fl}[2]{(-\Delta)^{#1}#2}
\newcommand{\RR}{\mathbb{R}}

\begin{document}

\begin{flushright}
V\'ictor Hern\'andez-Santamar\'ia
\\
Instituto de Matem\'aticas, UNAM
\\
Circuito Exterior, C.U., C.P. 04510
\\
CDMX, Mexico
\\
victor.santamaria@im.unam.mx
\\
\end{flushright}

\begin{flushleft}
Professor Mark Ainsworth
\\
Editor-in-Chief
\\
SIAM Journal on Numerical Analysis
\end{flushleft}

$\newline$

\begin{flushright}
November 21st, 2023
\end{flushright}

\noindent Dear Editor,
\\
\\
\indent Please find enclosed the manuscript of an original research work entitled “FEM for 1D-problems involving the logarithmic Laplacian: error estimates and numerical implementation”, which we are submitting for consideration in SIAM Journal on Numerical Analysis.

This is a joint work with Sven Jarohs and Leonard Sinsch from Institut f\"ur Mathematik at Goethe-Universit\"at Frankfurt and Alberto Salda\~{n}a of Instituto de Matem\'aticas at Universidad Nacional Autónoma de México.

In this paper, we develop a finite element method for one-dimensional problems involving the logarithmic Laplacian. This operator appears as a first-order expansion of the fractional Laplacian as the exponent $s$ goes to zero and exhibits new interesting phenomena as lack of good scaling properties, lack of maximum principles (in general), poor regularity of classical solutions, and lack of an exact formula of a torsion function. Using recently obtained regularity results, we provide a rigorous numerical analysis and  provide error estimates for the FEM showing a logarithmic order of convergence in the energy norm employing suitable \emph{log}-weighted spaces. We include a closed expression concerning the computation of the stiffness matrix of the logarithmic Laplacian as a derivative of the one of the fractional Laplacian, as well as a discussion regarding the discrete eigenvalue problem. 
 
\medskip

Thank you for considering our submission. %Please, address all the correspondence concerning this manuscript to me and feel free to contact me by email address.
\\
\\
\begin{flushright}
Sincerely yours,
\\
V\'ictor Hern\'andez-Santamar\'ia 
\end{flushright} 
\end{document}
